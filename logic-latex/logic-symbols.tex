% 中文文档类;如果用 article 等文档类,要额外设置中文字体,略麻烦
\documentclass[UTF8,12pt,a4paper]{ctexart}  


%% 各类数学宏包
\usepackage{amsmath}
\usepackage{amsthm}		%注意:If the amsthm package is used with a non-AMS document class and with the amsmath package, amsthm must be loaded after amsmath, not before.
\usepackage{amssymb}
\usepackage{mathtools}
\usepackage{amsfonts}
\usepackage{wasysym}

%% 设置页边距:
\usepackage{geometry}
\geometry{left=2cm,right=2cm,top=2cm,bottom=2cm}


\usepackage{graphicx} 	%管理图片的宏包

\usepackage{color}


%% 花体字母宏包
\usepackage{mathrsfs}

%% 自定义 「否定」 符号,使之与教材一致
\newcommand{\negs}{\sim\!\!}


%\\\\\\\\\\\\\\\\\\\\\\\\\\\\\\\\\\\\\\\\\\\\\\\\\\\\\\\\\\\\\\\\\\\\\\\\\\\\\\\
\begin{document}
	

\begin{center}
常用逻辑符号的 \LaTeX 代码 
\end{center}


\textit{本文档包含一些常用逻辑符号的 \LaTeX 代码,所列符号尽量与教材上的保持一致。此外,
本文档将会随课程进度持续更新。}

\textit{本文预设读者知道一些基本的 \LaTeX 操作,如怎样创建文档、怎样输入公式等。本文的内容只关乎逻辑符号和与之相关的注意事项。有任何技术上需要帮助的地方,请不要犹豫,联系助教。}

\textit{最后更新日期:\today}



\subsection*{命题联结词}

\begin{table}[h]
	\centering
	\caption{ 注意:以下代码必须在数学环境(行内/行间)中使用 } 
%	\label{tab: }
	\begin{tabular}{l l c l}
		\hline 
		\hline
		名称 & 代码 & 效果 & 示例\\
		\hline
		
		析取 & \textbackslash lor &  $\lor$ &$p \lor q$ \\
		
		合取 & \textbackslash land &  $\land$ &$p \land q$ \\
		
		蕴涵 & \textbackslash to &  $\to$ &$p \to q$ \\
		
		双蕴涵 & \textbackslash leftrightarrow &  $\leftrightarrow$ &$p \leftrightarrow q$ \\ 
		
		否定 & \textbackslash neg &  $\neg$ & $ \neg q $ \\
		
		否定(另一种写法) & \textbackslash sim \textbackslash! &  $\sim$ & $ \sim\! q $ \\
		\hline
		\hline
	\end{tabular}
\end{table}

注意事项:
\begin{itemize}
	\item \textbf{合取}:在一些“上古时代”的中英文教材中,也用符号 \& (其代码是 \textbackslash \&)表示合取,比如 $p \;\&\; q$。但因为该记法书写不方便,不建议使用这种记法。
	
	\item \textbf{蕴涵}:也有人会用代码 \textbackslash rightarrow 来表示蕴涵,但在 \LaTeX 中
	其和 \textbackslash to 所指的符号是同一个,所以为简单起见,用后者为佳。
	
	\item \textbf{否定}:教材上的否定符号$\sim$是一种老式的记法,但要注意的是,在 \LaTeX 中默认 \textbackslash sim 是一个二元关系符号,而否定是一个{\color{red} 一元}联结词。
	因此如果直接使用 \textbackslash sim 会导致 否定符号和公式之间的间距过宽,而上表中使用 \textbackslash sim \textbackslash! 则 缩短了 \textbackslash sim  的右间距,
	从而使之更像一个一元运算符。
	
	比较以下两者的效果:
	$\sim p$ 和 $\sim\! p$,
	前者的代码是 \textbackslash sim p  而后者的是 \textbackslash sim \textbackslash! p,会发现后面公式的否定符号和$p$之间的间距明显小于前一个公式。
	如果觉得其间距还是过大,可用 \textbackslash sim \textbackslash! \textbackslash! 继续调整,其效果是 $\sim\!\! p$。 一个更方便的做法是,在 tex 文件的导言区定义一个新的命令,如 \textbackslash newcommand \{\textbackslash oldneg \}\{\textbackslash sim \textbackslash!\},以后就可以直接用 \textbackslash oldneg 愉快地输入否定符号啦。
	
	总之,以上讨论只适用于强迫症者。如果觉得以上操作太麻烦了,推荐直接用 \textbackslash neg 完事。
\end{itemize}




















%\\\\\\\\\\\\\\\\\\\\\\\\\\\\\\\\\\\\\\\
\end{document}