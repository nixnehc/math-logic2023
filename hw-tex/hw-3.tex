% 中文文档类;如果用 article 等文档类,要额外设置中文字体,略麻烦
\documentclass[UTF8,12pt,a4paper]{ctexart}  


%% 各类数学宏包
\usepackage{amsmath}
\usepackage{amsthm}		%注意:If the amsthm package is used with a non-AMS document class and with the amsmath package, amsthm must be loaded after amsmath, not before.
\usepackage{amssymb}
\usepackage{mathtools}
\usepackage{amsfonts}
\usepackage{wasysym}

%% 设置页边距:
\usepackage{geometry}
\geometry{left=1cm,right=1cm,top=2cm,bottom=2cm}


\usepackage{graphicx} 	%管理图片的宏包

\usepackage{xcolor}


%% 花体字母宏包
\usepackage{mathrsfs}

%% 自定义 「否定」 符号,使之与教材一致
\newcommand{\negs}{\sim\!}


%\\\\\\\\\\\\\\\\\\\\\\\\\\\\\\\\\\\\\\\\\\\\\\\\\\\\\\
\begin{document}
	

\begin{center}
hw-3 (2023/09/26) \qquad\qquad 姓名:  \hspace{7em}  学号: 
\end{center}



\emph{p15}: 11-(a) \qquad
Show, using Proposition 1.14 and 1.17, that the statement form   \\
$(  (\neg (p \lor (\neg q)))  \to  (q \to r) )$
is logically equivalent to each of the following.

\begin{enumerate}
	\item[(a)] $ ( (\neg (q \to p)) \to ((\neg q) \lor r )) $
\end{enumerate}


\dotfill
\textit{Recall that}:
\dotfill

{\small 
	\textbf{Proposition 1.14}:
If $\mathscr{B}_1$ is a statement form arising from the statement form $\mathscr{A}$ 
by substituting the statement form $\mathscr{B}$ for one or more occurrences of the
statement form $\mathscr{A}$ in $\mathscr{A}_1$, 
and if $\mathscr{B}$ is logically equivalent to $\mathscr{A}$, 
then $\mathscr{B}_1$ is logically equivalent to $\mathscr{A}_1$.

	\textbf{Proposition 1.17 (De Morgan's Laws)}: 
Let $\mathscr{A}_1, \mathscr{A}_2, \dotsm \mathscr{A}_n$ be any statement forms. Then:
\begin{enumerate}
	\item $(\bigvee^n_{i=1} (\neg \mathscr{A}_i))$ is logically equivalent to $( \neg (\bigwedge^n_{i=1} \mathscr{A}_i))$.
	
	\item $(\bigwedge^n_{i=1} ( \neg \mathscr{A}_i))$  is logically equivalent to  $(\neg (\bigvee^n_{i=1} \mathscr{A}_i))$.
\end{enumerate}
}

\dotfill


\textbf{Your answer}:
















%\\\\\\\\\\\\\\\\\\\\\\\\\\\\\\\\\\\\\\\
\end{document}