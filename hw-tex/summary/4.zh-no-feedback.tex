% 中文文档类;如果用 article 等文档类,要额外设置中文字体,略麻烦
% 请用  XeLaTeX 编译!!!
\documentclass[UTF8,12pt,a4paper]{ctexart}  

%% 各类数学宏包
\usepackage{amsmath}
\usepackage{amsthm}		%注意:If the amsthm package is used with a non-AMS document class and with the amsmath package, amsthm must be loaded after amsmath, not before.
\usepackage{amssymb}
\usepackage{mathtools}
\usepackage{amsfonts}
\usepackage{wasysym}
\usepackage{fontawesome5}

%% 设置页边距:
\usepackage{geometry}
\geometry{left=1.5cm,right=1.5cm,top=1.5cm,bottom=1.5cm}

\usepackage{graphicx} 	%管理图片的宏包

\usepackage{xcolor}

%% 花体字母宏包
\usepackage{mathrsfs}

\usepackage{tcolorbox}
\tcbuselibrary{most}

%% 自定义 「否定」 符号,使之与教材一致
\newcommand{\negs}{\sim\!}

\begin{document}
%\\\\\\\\\\\\\\\\\\\\\\\\\\\\\\\\\\\\\\\\\\\\\\\\\\\\\\\\\\\\\\\\\\\\\\\\\\\\\\\

\begin{center}
	\textit{2023·秋·数理逻辑} \qquad \textit{平时作业汇总}  \qquad 大家辛苦啦!  \faCoffee ~ ~\\
	(\textit{中文参考答案} + \textit{不含作业反馈})
\end{center}


%=================== hw-1 
\noindent\texttt{hw-1 (2023/09/12) }

\emph{p3: 1-(h)} \quad
If $y$ is an integer then $z$ is not real, provided that $x$ is a rational number.


\noindent\textbf{Answer}:

令
\begin{center}
	\begin{tabular}{l l l }
 $p:$ & $y$ is an integer  \\
 $q:$ & $z$ is a real number  \\
 $r:$ & $x$ is a rational number 
\end{tabular}
\end{center}

\indent 
因此我们可得
 $r \to (p \to \neg q)$ or $(r \land p) \to \neg q$.
\hfill $\Box$







\vspace{3em}
%\\\\\\\\\\\\\\\\\\\\\\\\\\\\\\\\\\\\\\\\\\\\\\\\\\\\\\\\\\\\\\\\\\\\\\\\\\\\\\\
%\\\\\\\\\\\\\\\\\\\\\\\\\\\\\\\\\\\\\\\\\\\\\\\\\\\\\\\\\\\\\\\\\\\\\\\\\\\\\\\
%=================== hw-2 
\noindent\texttt{hw-2 (2023/09/19)}

\emph{p10}: (7) \quad
Show that the statement form $(((\negs p) \to q)  \to (p \to (\negs q)))$ 
is not a tautology. Find statement forms $\mathscr{A}$ and $\mathscr{B}$ such that 
$(((\negs \mathscr{A}) \to \mathscr{B})  \to (\mathscr{A} \to (\negs \mathscr{B})))$ is a contradiction.

\noindent\textbf{Answer}:

\noindent \texttt{方法一}:

下面的真值表表明公式
 $(((\negs p) \to q)  \to (p \to (\negs q)))$ 
\textit{不是}一个重言式:
\begin{center}
	\hspace{8em} 
	\begin{tabular}{@{ }c@{ }@{ }c | c@{ }@{}c@{}@{ }c@{ }@{ }c@{ }@{ }c@{ }@{ }c@{ }@{}c@{}@{ }c@{ }@{}c@{}@{ }c@{ }@{ }c@{ }@{ }c@{ }@{ }c@{ }@{}c@{}@{ }c}
		$p$ & $q$ &  & ( & $\lnot$ & $p$ & $\rightarrow$ & $q$ & ) & $\rightarrow$ & ( & $p$ & $\rightarrow$ & $\lnot$ & $q$ & ) & \\
		\hline 
		T & T &  &  & F & T & T & T &  & \textcolor{red}{F} &  & T & F & F & T &  & \\
		T & F &  &  & F & T & T & F &  & {T} &  & T & T & T & F &  & \\
		F & T &  &  & T & F & T & T &  & {T} &  & F & T & F & T &  & \\
		F & F &  &  & T & F & F & F &  & {T} &  & F & T & T & F &  & \\
	\end{tabular}
\end{center}

当  $\mathscr{A}$ 和 $\mathscr{B}$ 都是\textit{重言式}(tautology)时,
$(((\negs \mathscr{A}) \to \mathscr{B})  \to (\mathscr{A} \to (\negs \mathscr{B})))$ 
将变成一个矛盾式。
例如,让 $\mathscr{A} = \mathscr{B} = (p \to p)$ 抑或令
$\mathscr{A} = \mathscr{B} = (p \lor  \neg p)$。
\hfill $\Box$


\vspace{1em}
\noindent \texttt{方法二}:

(除了用真值表这种比较直观的手段外,还有诸多方法。以下答案来自\textit{黄程}同学,经其授权后分享给大家,感谢\textit{黄程}同学!)

假设 $(((\negs p) \to q)  \to (p \to (\negs q)))$ 是重言式。
那么在任意的\textit{赋值}(valuation)下,将永远不出现 $(\negs p) \to q $ 为 $T$ 且 $p \to (\negs q)$ 为 $F$ 的情况。 
但是如果令 $q = T$ 且 $p = T$, 
则 $p \to (\negs q)$ 的真值为 $T$。
矛盾! 
因此 $(((\negs p) \to q)  \to (p \to (\negs q)))$ 不是重言式。


根据上述回答,
当  $\mathscr{A}$ 和 $\mathscr{B}$ 永远为 $T$ 的时候,
$(((\negs \mathscr{A}) \to \mathscr{B})  \to (\mathscr{A} \to (\negs \mathscr{B})))$ 会是一个矛盾式。
换而言之,此时
$\mathscr{A}$ 和  $\mathscr{B}$ 都是重言式即可,
比如 
$\mathscr{A} = (p \lor  (\negs p))$ 且 $\mathscr{B} = p \to (q \to p)$。
\hfill $\Box$






\vspace{3em}
%\\\\\\\\\\\\\\\\\\\\\\\\\\\\\\\\\\\\\\\\\\\\\\\\\\\\\\\\\\\\\\\\\\\\\\\\\\\\\\\
%\\\\\\\\\\\\\\\\\\\\\\\\\\\\\\\\\\\\\\\\\\\\\\\\\\\\\\\\\\\\\\\\\\\\\\\\\\\\\\\
%=================== hw-3
\noindent\texttt{hw-3 (2023/09/26)}

\emph{p15}: 11-(a) \qquad
Show, using \textbf{Proposition 1.14} and \textbf{1.17}, that the statement form  
$(  (\neg (p \lor (\neg q)))  \to  (q \to r) )$
is logically equivalent to each of the following.

\hspace{1em} (a) $ ( (\neg (q \to p)) \to ((\neg q) \lor r )) $

\newtcolorbox{mybox}[2][]
{   colback = white, 
	colframe = black, 
	fonttitle = \bfseries,
	colbacktitle = gray, enhanced,
	attach boxed title to top center={yshift=-2mm},
	title=#2,#1}
\begin{mybox}[colback=white,width=19cm,boxrule=0.2mm]{Recall that}
	
	- \textbf{Proposition 1.14}:
	If $\mathscr{B}_1$ is a statement form arising from the statement form $\mathscr{A}$ 
	by substituting the statement form $\mathscr{B}$ for one or more occurrences of the
	statement form $\mathscr{A}$ in $\mathscr{A}_1$, 
	and if $\mathscr{B}$ is logically equivalent to $\mathscr{A}$, 
	then $\mathscr{B}_1$ is logically equivalent to $\mathscr{A}_1$.
	
	-  	\textbf{Proposition 1.17 (De Morgan's Laws)}: 
	Let $\mathscr{A}_1, \mathscr{A}_2, \dotsm \mathscr{A}_n$ be any statement forms. Then:
	\begin{enumerate}
		\item $(\bigvee^n_{i=1} (\neg \mathscr{A}_i))$ is logically equivalent to $( \neg (\bigwedge^n_{i=1} \mathscr{A}_i))$.
		
		\item $(\bigwedge^n_{i=1} ( \neg \mathscr{A}_i))$  is logically equivalent to  $(\neg (\bigvee^n_{i=1} \mathscr{A}_i))$.
	\end{enumerate}
\end{mybox}


\noindent\textbf{Answer}:
令
$\varphi = (  (\neg (p \lor (\neg q)))  \to  (q \to r) )$ 且  
$\chi = ( (\neg (q \to p)) \to ((\neg q) \lor r ))$.


据教材 \textbf{Prop. 1.14},我们只需要说明:
$\neg (p \lor (\neg q))$ \textit{逻辑等值}(logically equivalent)于  $(\neg (q \to p))$
且 $(q \to r)$ 逻辑等值于 $ (\neg q) \lor r )$, 
那么就有 $\varphi$ 逻辑等值 $\chi$ 。

不过很容易验证(比如说用\texttt{真值表}),
\[\begin{array}{r l l}
	\neg (p \lor (\neg q))  &\leftrightarrow&  (\neg (q \to p))  \qquad \text{ 和 }\\
	
	(q \to r) &\leftrightarrow&  (\neg q) \lor r )   
\end{array}\]
都是\textit{重言式},
这也意味着
 $(\neg (p \lor (\neg q)))$ 和 $(\neg (q \to p))$,
$(q \to r)$ 和 $(\neg q) \lor r ) $ 互相逻辑等值。 
\hfill $\Box$






\vspace{3em}
%\\\\\\\\\\\\\\\\\\\\\\\\\\\\\\\\\\\\\\\\\\\\\\\\\\\\\\\\\\\\\\\\\\\\\\\\\\\\\\\
%\\\\\\\\\\\\\\\\\\\\\\\\\\\\\\\\\\\\\\\\\\\\\\\\\\\\\\\\\\\\\\\\\\\\\\\\\\\\\\\
%=================== hw-4
\noindent\texttt{hw-4 (2023/10/10)}

\emph{p.19}: 13-(a) \quad
Find statement forms in \textbf{conjunctive normal form} which are logically equivalent to the following:
\[
(a) \qquad (((\neg p)  \lor q) \to r)
\] 

\noindent\textbf{Answer}: 
下面我们将用 3 种方法来寻找公式 $(\neg p \lor q) \to r$ 的\textit{合取范式} (\textbf{conjunctive normal forms},CNF),
前两种可以在教材上找的,而后一种是额外的补充内容。

不过首先,令 
\[
\varphi = (\neg p \lor q) \to r.
\]

\noindent\texttt{方法一}

首先我们画出公式 $\varphi$ {\color{red} 否定} (即$\neg\varphi$)的真值表:
\begin{center}
	\begin{tabular}{@{ }c@{ }@{ }c@{ }@{ }c | c@{ }@{}c@{}@{}c@{}@{ }c@{ }@{ }c@{ }@{ }c@{ }@{ }c@{ }@{}c@{}@{ }c@{ }@{ }c@{ }@{}c@{ }}
		$p$ & $q$ & $r$ & {\color{red} $\lnot$} & ( & ( & $\lnot$ & $p$ & $\lor$ & $q$ & ) & $\rightarrow$ & $r$ & )\\
		\hline 
		1 & 1 & 1 & 0 &  &  & 0 &  & 1 &  &  & 1 &  & \\
		\underline{1} & \underline{1} & \underline{0} & \textcolor{red}{1} &  &  & 0 &  & 1 &  &  & 0 &  & \\
		1 & 0 & 1 & 0 &  &  & 0 &  & 0 &  &  & 1 &  & \\
		1 & 0 & 0 & 0 &  &  & 0 &  & 0 &  &  & 1 &  & \\
		0 & 1 & 1 & 0 &  &  & 1 &  & 1 &  &  & 1 &  & \\
		\underline{0} & \underline{1} & \underline{0} & \textcolor{red}{1} &  &  & 1 &  & 1 &  &  & 0 &  & \\
		0 & 0 & 1 & 0 &  &  & 1 &  & 1 &  &  & 1 &  & \\
		\underline{0} & \underline{0} & \underline{0} & \textcolor{red}{1} &  &  & 1 &  & 1 &  &  & 0 &  & \\
	\end{tabular}
\end{center}

由上表可知,
使得 $\neg \varphi$ 为 1 的真值组合分别是
$110$、$010$ 以及 $000$。
因此  $\neg \varphi$ 的一个\textit{析取范式} 
(\textbf{disjunctive normal form})是
\[
\chi = (p \land q \land \neg r) \lor (\neg p \land q \land \neg r) \lor (\neg p \land  \neg q \land \neg r)
\] 
显然 $\chi$ 逻辑等值于 $\neg \varphi$,
因此 $\neg \chi $ 逻辑等值于 $ \neg \neg \varphi$,即 $\varphi$。

由\textit{德摩根律}(the \textbf{De Morgan’s laws})有
\[\begin{array}{l l l}
	\neg \chi 
	&=&   
	\neg [ (p \land q \land \neg r) \lor (\neg p \land q \land \neg r) \lor (\neg p \land  \neg q \land \neg r)  ] \\
	
	&\equiv&   
	{\color{red} \neg} (p \land q \land \neg r) \land {\color{red} \neg} (\neg p \land q \land \neg r) \land {\color{red} \neg} (\neg p \land  \neg q \land \neg r)  \\
	
	&\equiv&   
	( \neg p \lor \neg q \lor  \neg \neg r) \land  ( \neg \neg p \lor \neg q \lor \neg \neg r) \land  (\neg \neg p \lor \neg  \neg q \lor \neg \neg r)  \\
	
	&\equiv&   
	( \neg p \lor \neg q \lor  r) \land  ( p \lor \neg q \lor r) \land  ( p \lor  q \lor r)  \\
\end{array}\]

因此 $ ( \neg p \lor \neg q \lor  r) \land  ( p \lor \neg q \lor r) \land  ( p \lor  q \lor r) $ 是一个 $\varphi$ 的\textit{合取范式}。
\hfill $\Box$

(\textbf{注意}: 此处我们使用符号 “$\alpha \equiv \beta$” \; 来表示公式 $\alpha$ 和 $\beta$ 是逻辑等值的)



\vspace{1em}
\noindent\texttt{方法二}

\[\begin{array}{l l l l}
	\varphi 
	&=& 
	(\neg p \lor q \to r) \\
	
	&\equiv&
	\neg ( \neg p \lor q) \lor r  & \text{(由\textit{实质蕴含} \textbf{material implication}的含义,}   \text{ cf. p.7: Example 1.4-(a) )}\\
	
	&\equiv&
	(\neg \neg p \land \neg q) \lor r & \text{(由\textit{德摩根律}(the \textbf{De Morgan’s laws}))}  \\
	
	&\equiv&
	( p \land \neg q) \lor r   \\
	
	&\equiv&
	(p \lor r) \land (\neg q \lor r) & ( \lor\text{-}\land \text{ 间的分配 \;}, 
	\text{ cf. p.10, Exercises-6-(b)} )\\ 
\end{array}\]

因此 $(p \lor r) \land (\neg q \lor r)$ 是一个$\varphi$的合取范式。
\hfill $\Box$



\vspace{1em}

\noindent\texttt{方法三}

类似地,
我们画出 $\varphi$ 的真值表 (注意哟,不是 $\varphi$ \textit{否定} 的真值表):

\begin{center}
	\begin{tabular}{@{ }c@{ }@{ }c@{ }@{ }c | c@{ }@{}c@{}@{ }c@{ }@{ }c@{ }@{ }c@{ }@{ }c@{ }@{}c@{}@{ }c@{ }@{ }c@{ }@{ }c}
		$p$ & $q$ & $r$ &  & ( & $\lnot$ & $p$ & $\lor$ & $q$ & ) & $\rightarrow$ & $r$ & \\
		\hline 
		1 & 1 & 1 &  &  & 0 & 1 & 1 & 1 &  &                  1 & 1 & \\
		\underline{1} & \underline{1} & \underline{0} &  &  & 0 & 1 & 1 & 1 &  & \textcolor{red}{0} & 0 & \\
		1 & 0 & 1 &  &  & 0 & 1 & 0 & 0 &  &                 1  & 1 & \\
		1 & 0 & 0 &  &  & 0 & 1 & 0 & 0 &  & 				 1 & 0 & \\
		0 & 1 & 1 &  &  & 1 & 0 & 1 & 1 &  & 				 1 & 1 & \\
		\underline{0} & \underline{1} & \underline{0} &  &  & 1 & 0 & 1 & 1 &  & \textcolor{red}{0} & 0 & \\
		0 & 0 & 1 &  &  & 1 & 0 & 1 & 0 &  & 				 1 & 1 & \\
		\underline{0} & \underline{0} & \underline{0} &  &  & 1 & 0 & 1 & 0 &  & \textcolor{red}{0} & 0 & \\
	\end{tabular}
\end{center}

令 $\varphi$ 为 $0$ 的真值组合分别是
$110$、$010$ 和 $000$。
随后根据这些真值组合,可以构造出如下 3 个 析取公式:
\[\begin{array}{l l l}
	\varphi_1 
	&=&
	(\neg p \lor \neg q \lor r) \\ 
	
	\varphi_2
	&=&
	( p \lor \neg q \lor r) \\
	
	\varphi_3
	&=&
	( p \lor  q \lor r)  \\
\end{array}\]

接下来,我们将上面这三个公式合取起来,
\[
\varphi_1 \land \varphi_2 \land \varphi_3 = ( \neg p \lor \neg q \lor  r) \land  ( p \lor \neg q \lor r) \land  ( p \lor  q \lor r)
\]

容易验证  $\varphi_1 \land \varphi_2 \land \varphi_3$ 是一个 $\varphi$ 的合取范式。
\hfill $\Box$

[ps. 正如我们所见,\texttt{方法一}和\texttt{方法三}所得的合取范式是相同的] 

%%%%%%%%%%%%%%%%%%%%%%%%%%%%%%%%%%%%%%%%%%%%%%%%%%%%%%%%%%%%%%%%%%%%%%%%%%%%%%%%
%\vspace{2em}  % 如果用 latex 答题 请把这个垂直间距删除,或调整为自己喜欢的间距
\vspace{3em}

\emph{p.26}: 21 \quad
Suppose that $\mathscr{A}_1,\mathscr{A}_2,\dots,\mathscr{A}_n$;  $\therefore\mathscr{A}$ is a valid argument form. Prove that $\mathscr{A}_1,\mathscr{A}_2,\dots,\mathscr{A}_{n-1}$;  $\therefore (\mathscr{A}_n \to \mathscr{A})$ is also a valid argument form.

\noindent\textbf{Proof}:

首先,假设
$\mathscr{A}_1,\mathscr{A}_2,\dots,\mathscr{A}_n$;  $\therefore\mathscr{A}$ 
是\textit{有效的}(valid)论证形式,但 $\mathscr{A}_1,\mathscr{A}_2,\dots,\mathscr{A}_{n-1}$;  $\therefore (\mathscr{A}_n \to \mathscr{A})$ 不是。 

那么存在一个真值指派,使得
$\mathscr{A}_1,\mathscr{A}_2,\dots,\mathscr{A}_{n-1}$ 为 $T$ 
而 $(\mathscr{A}_n \to \mathscr{A})$ 为 $F$,即 
$\mathscr{A}_n $ 为 $T$ 且 $\mathscr{A}$ 为 $F$。
然而,
这同我们的假设 ----- $\mathscr{A}_1,\mathscr{A}_2,\dots,\mathscr{A}_n$;  $\therefore\mathscr{A}$ 是有效的论证形式 ----- 矛盾!
\hfill $\Box$








\vspace{4em}
%\\\\\\\\\\\\\\\\\\\\\\\\\\\\\\\\\\\\\\\\\\\\\\\\\\\\\\\\\\\\\\\\\\\\\\\\\\\\\\\
%\\\\\\\\\\\\\\\\\\\\\\\\\\\\\\\\\\\\\\\\\\\\\\\\\\\\\\\\\\\\\\\\\\\\\\\\\\\\\\\
%=================== hw-5
\noindent\texttt{hw-5 (2023/10/17)}

\emph{p.36}: 1-(c) \quad
Write out proofs in $L$ for the following $wfs$.
\[
(c) \qquad  (p_1 \to (p_1 \to p_2)) \to (p_1 \to p_2)
\] 


\noindent\textbf{Proof}: 

\noindent\texttt{方法一}

\begin{enumerate}
	\item $ (p_1 \to (p_1 \to p_2)) \to ((p_1 \to p_1) \to (p_1 \to p_2)) $
	\hfill (instance of $L2$)
	
	\item $ [(p_1 \to (p_1 \to p_2)) \to ((p_1 \to p_1) \to (p_1 \to p_2))] \to  $
	
	$
	[( (p_1 \to (p_1 \to p_2)) \to (p_1 \to p_1))  \to
	( (p_1 \to (p_1 \to p_2)) \to (p_1 \to p_2)) ]
	$
	\hfill (instance of $L2$)
	
	\item $ ( (p_1 \to (p_1 \to p_2)) \to (p_1 \to p_1))  \to
	( (p_1 \to (p_1 \to p_2)) \to (p_1 \to p_2))$
	\hfill ($1+2,MP$)
	
	\item $p_1 \to ((p_1 \to p_2) \to p_1)$ 
	\hfill (instance of $L1$)
	
	\item $[p_1 \to ((p_1 \to p_2) \to p_1)] \to 
	[ (p_1 \to (p_1 \to p_2))  \to (p_1 \to p_1) ]$ 
	\hfill (instance of $L2$)
	
	\item $(p_1 \to (p_1 \to p_2))  \to (p_1 \to p_1)$
	\hfill ($4+5, MP$)
	
	\item $(p_1 \to (p_1 \to p_2)) \to (p_1 \to p_2)$ 
	\hfill ($3+6, MP$)
\end{enumerate}
当然 (c) 的证明不是唯一的。
\hfill $\Box$


%---------------------------------------
\vspace{1em} 

\noindent\texttt{方法二}

\begin{enumerate}
	\item $p_1 \to ((p_1 \to p_1) \to p_1)$  
	\hfill (instance of $L1$)
	
	\item $(p_1 \to ((p_1 \to p_1) \to p_1))   \to   (  (p_1 \to (p_1 \to p_1)) \to (p_1 \to p_1) )$ 
	\hfill (instance of $L2$)
	
	\item $(p_1 \to (p_1 \to p_1)) \to (p_1 \to p_1)$ 
	\hfill ($1+2, MP$)
	
	\item $p_1 \to (p_1 \to p_1)$ 
	\hfill (instance of $L1$)
	
	\item $(p_1 \to p_1)$
	\hfill ($3+4, MP$)
	
	\item $(p_1 \to p_1) \to ( (p_1 \to (p_1 \to p_2)) \to (p_1 \to p_1) )$  
	\hfill (instance of $L1$)
	
	\item $(p_1 \to (p_1 \to p_2)) \to (p_1 \to p_1)$ 
	\hfill ($5+6, MP$)
	
	\item $( p_1 \to (p_1 \to p_2))  \to  ( (p_1 \to p_1)  \to (p_1 \to p_2) )$  
	\hfill (instance of $L2$)
	
	\item $ [(p_1 \to (p_1 \to p_2)) \to ((p_1 \to p_1) \to (p_1 \to p_2))] \to  $
	
	$
	[( (p_1 \to (p_1 \to p_2)) \to (p_1 \to p_1))  \to
	( (p_1 \to (p_1 \to p_2)) \to (p_1 \to p_2)) ]
	$
	\hfill (instance of $L2$)
	
	\item $( (p_1 \to (p_1 \to p_2)) \to (p_1 \to p_1))  \to
	( (p_1 \to (p_1 \to p_2)) \to (p_1 \to p_2)) $  
	\hfill ($8+9,MP$)
	
	\item $(p_1 \to (p_1 \to p_2)) \to (p_1 \to p_2)$ 
	\hfill ($7+10, MP$)
\end{enumerate}


%---------------------------------------
\vspace{1em}

\noindent\texttt{方法三}

\begin{enumerate}
	\item  $ \{ (p_1 \to p_2) \to [ ( (p_1 \to p_2) \to (p_1 \to p_2) )  \to (p_1 \to p_2)   ] \} \to $
	
	$ \{ [ (p_1 \to p_2) \to ( (p_1 \to p_2) \to (p_1 \to p_2) ) ] \to [(p_1 \to p_2) \to (p_1 \to p_2)]  \}$
	\hfill (instance of $L2$)
	
	
	\item $(p_1 \to p_2) \to [ ( (p_1 \to p_2) \to (p_1 \to p_2) )  \to (p_1 \to p_2)   ]$
	\hfill (instance of $L1$)
	
	\item $[ (p_1 \to p_2) \to ( (p_1 \to p_2) \to (p_1 \to p_2) ) ] \to [(p_1 \to p_2) \to (p_1 \to p_2)] $ 
	\hfill ($1+2, MP$)
	
	\item $(p_1 \to p_2) \to ( (p_1 \to p_2) \to (p_1 \to p_2) ) $ 
	\hfill (instance of $L1$)
	
	\item $(p_1 \to p_2) \to (p_1 \to p_2)$  
	\hfill ($3+4, MP$)
	
	\item $[ (p_1 \to p_2) \to (p_1 \to p_2)]   \to [ ((p_1 \to p_2) \to p_1)  \to ((p_1 \to p_2) \to p_2)  ]$ 
	\hfill (instance of $L2$)
	
	\item $((p_1 \to p_2) \to p_1)  \to ((p_1 \to p_2) \to p_2)$ 
	\hfill ($5+6, MP$)
	
	\item $ [ ((p_1 \to p_2) \to p_1)  \to ((p_1 \to p_2) \to p_2) ]  \to $ 
	
	$
	[ p_1 \to ( ((p_1 \to p_2) \to p_1)  \to ((p_1 \to p_2) \to p_2) )  ]$
	\hfill (instance of $L1$)
	
	\item $p_1 \to ( ((p_1 \to p_2) \to p_1)  \to ((p_1 \to p_2) \to p_2) )  $ 
	\hfill ($7+8, MP$)
	
	\item $ [p_1 \to ( ((p_1 \to p_2) \to p_1)  \to ((p_1 \to p_2) \to p_2) )   ] \to$
	
	$[ (p_1 \to ((p_1 \to p_2) \to p_1))   \to (p_1 \to ((p_1 \to p_2) \to p_2))   ]$
	\hfill (instance of $L2$)
	
	\item $(p_1 \to ((p_1 \to p_2) \to p_1))   \to (p_1 \to ((p_1 \to p_2) \to p_2))  $  
	\hfill ($9+10,MP$)
	
	\item $p_1 \to ((p_1 \to p_2) \to p_1)$  
	\hfill (instance of $L1$)
	
	\item $ p_1 \to ((p_1 \to p_2) \to p_2)$  
	\hfill ($11+12, MP$)
	
	\item $[ p_1 \to ((p_1 \to p_2) \to p_2)]  \to [ (p_1 \to (p_1 \to p_2)) \to (p_1 \to p_2) ]$
	\hfill (instance of $L2$) 
	
	\item  $(p_1 \to (p_1 \to p_2)) \to (p_1 \to p_2)$
	\hfill ($13+14, MP$) 
\end{enumerate}


\vspace{1em}

(ps. 上面公式中的 \textit{中括号} $[ \;]$ 和 \textit{花括号} $\{ \; \}$ 是起辅助作用的,为的是方便大家观看。但应注意的是,其本身不是命题逻辑公理系统$L$中的符号 !!!)

\vspace{2em}

\emph{p.37}: 5 \quad
The rule $HS$ is an example of a legitimate additional rule of deduction for $L$. Is the following rule legitimate in the same sense: from the $wfs.$ $\mathscr{B}$ and $(\mathscr{A} \to (\mathscr{B} \to \mathscr{C}))$, deduce $(\mathscr{A} \to \mathscr{C})$ ?

\noindent\textbf{Answer}:   

\noindent\texttt{方法一} (不用\textbf{演绎定理}  (Deduction Theorem) )

\begin{enumerate}
	\item $\mathscr{B}$
	\hfill (假设)
	
	\item $(\mathscr{A} \to (\mathscr{B} \to \mathscr{C}))$
	\hfill (假设)
	
	
	\item $(\mathscr{A} \to (\mathscr{B} \to \mathscr{C})) \to ((\mathscr{A} \to \mathscr{B}) \to (\mathscr{A} \to \mathscr{C})) $
	\hfill ($L2$)
	
	
	\item $( (\mathscr{A} \to \mathscr{B}) \to (\mathscr{A} \to \mathscr{C}) )$
	\hfill ($2 + 3, MP$)
	
	\item $(\mathscr{B} \to (\mathscr{A} \to \mathscr{B}))$
	\hfill ($L1$)
	
	\item $(\mathscr{A} \to \mathscr{B})$
	\hfill ($1+5, MP$)
	
	\item $(\mathscr{A} \to \mathscr{C})$
	\hfill ($6+4, MP$)
\end{enumerate}

因此该规则对于系统$L$来说是合法的。
\hfill $\Box$



%---------------------------------------------------------
\vspace{2em}
\noindent\texttt{方法二} (使用\textbf{演绎定理}  (Deduction Theorem) )

首先我们表明
\[
\{\mathscr{B}, (\mathscr{A} \to (\mathscr{B} \to \mathscr{C}))\} \cup \{ \mathscr{A}\} \vdash_L \mathscr{C}.
\]
下面是其一个演绎:
\begin{enumerate}
	\item $\mathscr{B}$  
	\hfill (假设)
	
	\item $(\mathscr{A} \to (\mathscr{B} \to \mathscr{C}))$
	\hfill (假设)
	
	\item $\mathscr{A}$
	\hfill (假设)
	
	\item $(\mathscr{B} \to \mathscr{C})$
	\hfill ($2+3,MP$)
	
	\item $\mathscr{C}$
	\hfill ($1+4, MP$)
\end{enumerate} 

因此,由 \textbf{演绎定理}可知
$
\{\mathscr{B}, (\mathscr{A} \to (\mathscr{B} \to \mathscr{C}))\} \vdash_L \mathscr{A} \to \mathscr{C}.
$
\hfill $\Box$

\vspace{1em}








\vspace{4em}
%\\\\\\\\\\\\\\\\\\\\\\\\\\\\\\\\\\\\\\\\\\\\\\\\\\\\\\\\\\\\\\\\\\\\\\\\\\\\\\\
%\\\\\\\\\\\\\\\\\\\\\\\\\\\\\\\\\\\\\\\\\\\\\\\\\\\\\\\\\\\\\\\\\\\\\\\\\\\\\\\
%=================== hw-6
\noindent\texttt{hw-6 (2023/10/31)}  \textbf{期中作业}

\emph{p.44}: (8) \quad
Let $\mathscr{A}$ be a \textit{wf.} $((\neg p_1 \to p_2) \to (p_1 \to \neg p_2))$. 
Show that $L^+$, obtained by including this $\mathscr{A}$ as a new axiom, has a larger set of theorems than $L$. Is $L^+$ a consistent extension of $L$?
(\textit{注意:此题有两问})

\noindent\textbf{Proof}:

\begin{center}
	\begin{tabular}{@{ }c@{ }@{ }c | c@{ }@{}c@{}@{ }c@{ }@{ }c@{ }@{ }c@{ }@{ }c@{ }@{}c@{}@{ }c@{ }@{}c@{}@{ }c@{ }@{ }c@{ }@{ }c@{ }@{ }c@{ }@{}c@{}@{ }c}
		$p_1$ & $p_2$ &  & ( & $\lnot$ & $p_1$ & $\to$ & $p_2$ & ) & $\to$ & ( & $p_1$ & $\to$ & $\lnot$ & $p_2$ & ) & \\
		\hline 
		T & T &  &  & F & T & T & T &  & \textcolor{purple}{F} &  & T & F & F & T &  & \\
		T & F &  &  & F & T & T & F &  & \textcolor{purple}{T} &  & T & T & T & F &  & \\
		F & T &  &  & T & F & T & T &  & \textcolor{purple}{T} &  & F & T & F & T &  & \\
		F & F &  &  & T & F & F & F &  & \textcolor{purple}{T} &  & F & T & T & F &  & \\
	\end{tabular}
\end{center}

\texttt{一}:

据上面的真值表,显然 $\mathscr{A} = ((\neg p_1 \to p_2) \to (p_1 \to \neg p_2))$ 不是重言式。
因此由 \textbf{可靠性}(\textbf{Soundness Theorem}), 
$\mathscr{A}$  {\color{purple} 不是} $L$ 的\textit{定理}(theorem), 
而它却是 $L^+$ 的定理。
因此 $L^+$ 的定理集比 $L$ 的大。

\texttt{二}:

$L^+$ 是\textit{一致的}(consistent)。
假设 $L^+$ 不一致,则存在公式 $\mathscr{B}$ 使得 
$\vdash_{L^+} \mathscr{B}$  且	$\vdash_{L^+} \neg \mathscr{B}$。
因为 $L^+$ 是在 $L$ 的基础上添加额外的公理 $\mathscr{A} = ((\neg p_1 \to p_2) \to (p_1 \to \neg p_2))$ 而得到的,
因此可得(注意$\vdash$的{\color{purple} 下标})
\[\begin{array}{l l l}
	\mathscr{A} \vdash_{L} \mathscr{B}  \quad \text{and} \quad	
	\mathscr{A} \vdash_{L} \neg \mathscr{B}. \\
\end{array}\]
由\textbf{演绎定理}(\textbf{Deduction Theorem}),
\[
\vdash_{L} \mathscr{A} \to \mathscr{B}  \quad \text{and} \quad	
\vdash_{L} \mathscr{A} \to \neg \mathscr{B}, 
\]
由\textbf{可靠性}(\textbf{Soundness Theorem}),这意味着
($\mathscr{A} \to \mathscr{B})$ 和 $(\mathscr{A} \to \neg \mathscr{B})$ 都是重言式。
由定义,对任意的赋值$v$,
$v(\mathscr{A} \to \mathscr{B})  = T$ 且  $v(\mathscr{A} \to \neg \mathscr{B})  = T$,
这表明
$v(\mathscr{A}) = F$,即 $\mathscr{A}$ 是\textbf{矛盾式} (\textit{contradiction})。
但由上面  $\mathscr{A}$ 的真值表我们知道这是不可能的。
矛盾!
\hfill $\Box$



%%%%%%%%%%%%%%%%%%%%%%%%%%%%%%%%%%%%%%%%%%%%%%%%%%%%%%%%%%%%%%%%%%%%%%%%%%%%%%%%
\vspace{3em}  
\noindent\emph{p.44}: (10) \quad
Let $L^{++}$ be the extension of $L$ obtained by including as a fourth axiom \textit{scheme}:
\[
( (\neg \mathscr{A} \to \mathscr{B}) \to (\mathscr{A} \to \neg \mathscr{B})).
\]
Show that $L^{++}$ is inconsistent. (Hint: see Chapter 1 exercise 7 (p.10)) 


\noindent\textbf{Proof}:

\noindent\texttt{方法一}

令 
$\top = (p \to p)$ 且 $\varphi = (\neg \top \to \top)  \to (\top \to \neg \top)$。
显然
$ \vdash_{L^{++}} \varphi$ (即令 $\mathscr{A} = \mathscr{B} = \top$). 
容易验证, $\varphi$ 是一个矛盾式,因此 $\neg \varphi$ 是重言式。
由 \textbf{完全性}(\textbf{Completeness Theorem}), $\vdash_L \neg \varphi$, 
因为$L^{++}$ 是一个 $L$ 的扩张,因此 $\vdash_{L^{++}} \neg \varphi$。

但此时我们同时有 $\vdash_{L^{++}} \varphi$ 且 $\vdash_{L^{++}} \neg \varphi$,
据定义,$L^{++}$  不一致。
\hfill $\Box$


\vspace{1em}
\noindent\texttt{方法二}

(下面这个证明来自 \textit{吴家儒}\ 同学,这种证明很直接且颇具暴力美学,
再次感谢\textit{家儒}同学为我们带来如此精彩的证明!)

因为 $\vdash_L (p \to p)$ (参见 \textit{Example 2.7-(a)} in page 31),故
$ \vdash_{L^{++}} (p \to p) $。
令 $(L4)$ 表示 $L^{++}$ 的\textit{第四条公式模式}(\textit{the fourth axiom scheme}), 即
\[
(L4) \qquad ( (\neg \mathscr{A} \to \mathscr{B}) \to (\mathscr{A} \to \neg \mathscr{B})).
\]


考虑如下 $L^{++}$ 中的证明:
\begin{enumerate}
	\item $[\neg (p \to p) \to (p \to p)]   \to [(p \to p) \to \neg (p \to p)]$ 
	\hfill  ($L4$的实例)
	
	\item $[ (\neg (p \to p) \to (p \to p))   \to ((p \to p) \to \neg (p \to p)) ]$ $\to$
	
	$[
	((\neg (p \to p) \to (p \to p)) \to (p \to p) )  \to 
	(
	(\neg (p \to p) \to (p \to p)) \to \neg (p \to p)
	)
	]$
	\hfill  ($L2$的实例)
	
	\item $((\neg (p \to p) \to (p \to p)) \to (p \to p) )  \to 
	(
	(\neg (p \to p) \to (p \to p)) \to \neg (p \to p)
	)$
	\hfill ($1 + 2, MP$)
	
	\item $(p \to p) \to [(\neg (p \to p) \to (p \to p)) \to (p \to p)]$ 
	\hfill ($L1$的实例)
	
	\item $(p \to p) $ 
	\hfill ($(p\to p)$ 是 $L^{++}$的定理)
	
	\item $(\neg (p \to p) \to (p \to p)) \to (p \to p) $
	\hfill ($4+5, MP$)
	
	\item $(\neg (p \to p) \to (p \to p)) \to \neg (p \to p)$
	\hfill ($6+3, MP$)
	
	\item $(p \to p) \to ( \neg (p \to p) \to (p \to p))$ 
	\hfill ($L1$的实例)
	
	\item $ \neg (p \to p) \to (p \to p)$ 
	\hfill ($5+8, MP$)
	
	\item $\neg (p \to p)$
	\hfill ($9+7, MP$)
\end{enumerate}

\noindent 
因此 $\vdash_{L^{++}} \neg (p \to p)$,这和  $\vdash_{L^{++}} (p \to p)$ 共同说明了 $L^{++}$ 是不一致的。
\hfill $\Box$



\vspace{1em}
\dotfill hw-6: feedback
\dotfill





\vspace{4em}
%\\\\\\\\\\\\\\\\\\\\\\\\\\\\\\\\\\\\\\\\\\\\\\\\\\\\\\\\\\\\\\\\\\\\\\\\\\\\\\\
%\\\\\\\\\\\\\\\\\\\\\\\\\\\\\\\\\\\\\\\\\\\\\\\\\\\\\\\\\\\\\\\\\\\\\\\\\\\\\\\
%=================== hw-7
\noindent\texttt{hw-7 (2023/11/07)}

\emph{p.49}: 2-(c) \quad
Translate each of the following statements into symbols, {\color{purple} first} using no existential quantifiers, and {\color{purple} second} using no universal quantifiers. 

\qquad (c) \hspace{3cm}  \textit{No mouse is heavier than any elephant.}

(\textit{注意:题目要求大家要分别用“全称量词”和“存在量词”符号化句子,因此你的翻译至少有两句})

\noindent\textbf{Answer}:   

令 
\[\begin{array}{l l l}
	M(x):& x \text{ is a \textit{mouse}}\\
	E(x): & x \text{ is an \textit{elephant}}\\
	H(x,y): & x \text{ is \textit{heavier than}\;} y \\
\end{array}\]

不使用\textbf{存在量词}(existential quantifier): 
\begin{enumerate}
	\item $(\forall x) (\forall y) (M(x) \land E(y) \to \neg H(x,y))$, 或
	
	\item $(\forall x) (\forall y) (M(x) \to (E(y) \to \neg H(x,y)))$, 或
	
	\item $(\forall x)  (M(x) \to (\forall y) (E(y) \to \neg H(x,y)))$, 或
	
	\item 其余任何合理的答案。
\end{enumerate}

\vspace{1em}

不使用\textbf{全称量词}(universal quantifier):
\begin{enumerate}
	\item $\neg (\exists x) (\exists y) (M(x) \land E(y) \land H(x,y))$, 或
	
	\item $\neg (\exists x)  (M(x) \land (\exists y) (E(y) \land H(x,y)))$, 或
	
	\item 其余任何合理的答案。
	\hfill $\Box$
\end{enumerate}








\vspace{4em}
%\\\\\\\\\\\\\\\\\\\\\\\\\\\\\\\\\\\\\\\\\\\\\\\\\\\\\\\\\\\\\\\\\\\\\\\\\\\\\\\
%\\\\\\\\\\\\\\\\\\\\\\\\\\\\\\\\\\\\\\\\\\\\\\\\\\\\\\\\\\\\\\\\\\\\\\\\\\\\\\\
%=================== hw-8
\noindent\texttt{hw-8 (2023/11/14)}

\emph{p.56}: 9-(d) \quad
In each case below, let $\mathscr{A}(x_1)$ be the given \emph{wf}., and let $t$ be the term $f^2_1(x_1,x_3)$. Write out the $wf.$ $\mathscr{A}(t)$ and hence decide in each case whether $t$ is {\color{purple} free for $x_1$} in the given \emph{wf}.
\[
(d) \qquad (\forall x_2) A^3_1 (x_1, f^1_1(x_1),x_2)  \to (\forall x_3)A^1_1(f^2_1(x_1,x_3)).
\]

\newtcolorbox{mybox2}[2][]
{   colback = white, 
	colframe = black, 
	fonttitle = \bfseries,
	colbacktitle = gray, enhanced,
	attach boxed title to top center={yshift=-2mm},
	title=#2,#1}
\begin{mybox2}[colback=white,width=19cm,boxrule=0.2mm]{Recall that}
	
	- {\color{red} $\mathscr{A}(t)$}: if $x_i$ does occur free in $\mathscr{A}(x_1)$, then $\mathscr{A}(t)$ denotes the result of substituting term $t$ for  {\color{purple}every free occurrence} of $x_i$. (cf. \emph{p.54})	
	
	-  $t$ is {\color{red} free} for $x$ in a \emph{wf.} $\phi$: 
	
	\textbf{定义3.11*. (Revised defintion)} \quad
	当一个项 $t$ 可以替换 $\mathscr{A}$ 中变元 $x_i$ 的{\color{purple} 所有自由出现},且不会使得 $t$ 中任何变元与 $\mathscr{A}$ 的其他部分相互作用,我们就称 { \color{red} $t$ 对 $\mathscr{A}$ 中 $x_i$ 是自由的}。
\end{mybox2}


(\textit{注意此题有两问: 你需要 $1)$ 写出$\mathscr{A}(t)$,且 $2)$ 回答$t$在$\mathscr{A}(x_1)$中是否对$x_1$自由})

\noindent\textbf{Answer}:  

注意到在
\[
(d) \qquad (\forall x_2) A^3_1 ({\color{purple} x_1}, f^1_1({\color{purple} x_1}),x_2)  \to (\forall x_3)A^1_1(f^2_1({\color{purple} x_1},x_3)).
\]
中 $x_1$ 有 \textit{三处} 出现是\textit{自由的}(free),因此 
\[
\mathscr{A}(t) = (\forall x_2) A^3_1 ({\color{purple} f^2_1(x_1,x_3)}, f^1_1({\color{purple} f^2_1(x_1,x_3)}),x_2)  \to (\forall x_3)A^1_1(f^2_1({\color{purple} f^2_1(x_1,x_3)},x_3))  
\]
显然 $t$ 对 $(d)$ 中的 $x_1$ \textit{不是}自由的。 
\hfill $\Box$






\vspace{4em}
%\\\\\\\\\\\\\\\\\\\\\\\\\\\\\\\\\\\\\\\\\\\\\\\\\\\\\\\\\\\\\\\\\\\\\\\\\\\\\\\
%\\\\\\\\\\\\\\\\\\\\\\\\\\\\\\\\\\\\\\\\\\\\\\\\\\\\\\\\\\\\\\\\\\\\\\\\\\\\\\\
%=================== hw-9
\noindent\texttt{hw-9 (2023/11/21)}

\emph{p.59}: 11 \quad
Let $\mathscr{L}$ be the first order language which includes (besides variables, punctuation, 
connectives and quantifier) the individual constant $a_1$, the function letter $f^2_1$ and the predicate letter $A^2_2$. 
Let $\mathscr{A}$ denote the \textit{wf}.
\[
(\forall x_1)(\forall x_2)( A^2_2 ( f^2_1(x_1,x_2), a_1)  \to A^2_2 (x_1,x_2) ).
\]
Define an interpretation $I$ of $\mathscr{A}$ as follows. 
$D_I$ is $\mathbb{Z}$, $\bar{a}_1$ is $0$, $\bar{f^2_1} (x, y)$ is $x-y$, 
$\bar{A^2_2}(x,y)$ is $x < y$. 
Write down the interpretation of $\mathscr{A}$ in \textit{I}. 
Is this a true statement or a false one? 
Find another interpretation in which $\mathscr{A}$ is interpreted by a statement with the opposite truth value. \\
(\textit{注意此题有三问: $1)$ 用自然语言(中文/英语)写出 $\mathscr{A}$ 在 $I$下的直观含义;$2)$ 回答在\textit{I}下 $\mathscr{A}$是为\textbf{{\color{purple} 真}}还是为\textbf{{\color{purple} 假}};$3)$ 基于你对第二问的回答,为公式 $\mathscr{A}$ 找一个新的解释,且在这个{\color{purple} 新}解释中,$\mathscr{A}$的真值与你第二问的答案恰好相反})[所以你对第二问的回答很重要]

\noindent\textbf{Answer}:   

(1) 公式 $\mathscr{A}$ 在解释 $I$ 中的直观含义如下: 

\hspace{6em} \textit{对于任意整数 $x_1, x_2$: 如果  $(x_1 - x_2) < 0$ 那么 $x_1 < x_2$.}


(2) 上面对 $\mathscr{A}$ 在 $I$ 中的解释显然是\textbf{真的}({\color{purple} \textit{true}})。 


(3) 令 $D_I = \mathbb{N}$, $\bar{a}_1$ 指代 $0$, $\bar{f^2_1} (x, y)$ 意指 $x \times y$,  $\bar{A^2_2}(x,y)$ 是 $x > y$。
自然,$\mathscr{A}$ 在这个新解释中为\textbf{假}({\color{purple} false})。
 
[当然,其他任何合理的解释都是可接受的]
\hfill $\Box$













\vspace{4em}
%\\\\\\\\\\\\\\\\\\\\\\\\\\\\\\\\\\\\\\\\\\\\\\\\\\\\\\\\\\\\\\\\\\\\\\\\\\\\\\\
%\\\\\\\\\\\\\\\\\\\\\\\\\\\\\\\\\\\\\\\\\\\\\\\\\\\\\\\\\\\\\\\\\\\\\\\\\\\\\\\
%=================== hw-10
\noindent\texttt{hw-10 (2023/12/13)}

\emph{p.70}: 22-(a) \quad
Show that none of the following \textit{wfs.} is logically valid.
\[
(a) \qquad 
(\forall x_1) (\exists x_2) A^2_1(x_1,x_2) \to (\exists x_2) (\forall x_1) A^2_1 (x_1,x_2).
\]
\textbf{Proof}:   

只需要找到一个翻译 $I$ 使得 $I \not \models (\forall x_1) (\exists x_2) A^2_1(x_1,x_2) \to (\exists x_2) (\forall x_1) A^2_1 (x_1,x_2)$ 即可。

令 $D_I=\mathbb{N}$, $\bar{A}^2_1(x,y)$ 表示 `$x < y$'。


显然\textit{闭公式}  (close \textit{wf.}) $(\forall x_1) (\exists x_2) A^2_1(x_1,x_2)$ 在这个解释中为\textbf{真},而 
$(\exists x_2) (\forall x_1) A^2_1 (x_1,x_2)$ 在这个解释中为\textbf{假}。
因此,每个满足前件的赋值都不满足后件。
因此 不存在解释$I$上的赋值满足 $(a)$ 中的公式。
进而该公式不是逻辑有效的。 
\hfill $\Box$







%\\\\\\\\\\\\\\\\\\\\\\\\\\\\\\\\\\\\\\\\\\\\\\\\\\\\\\\\\\\\\\\\\\\\\\\\\\\\\\\
\end{document}