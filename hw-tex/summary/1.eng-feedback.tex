% 中文文档类;如果用 article 等文档类,要额外设置中文字体,略麻烦
% 请用  XeLaTeX 编译!!!
\documentclass[UTF8,12pt,a4paper]{ctexart}  

%% 各类数学宏包
\usepackage{amsmath}
\usepackage{amsthm}		%注意:If the amsthm package is used with a non-AMS document class and with the amsmath package, amsthm must be loaded after amsmath, not before.
\usepackage{amssymb}
\usepackage{mathtools}
\usepackage{amsfonts}
\usepackage{wasysym}
\usepackage{fontawesome5}

%% 设置页边距:
\usepackage{geometry}
\geometry{left=1.5cm,right=1.5cm,top=1.5cm,bottom=1.5cm}

\usepackage{graphicx} 	%管理图片的宏包

\usepackage{xcolor}

%% 花体字母宏包
\usepackage{mathrsfs}

\usepackage{tcolorbox}
\tcbuselibrary{most}

%% 自定义 「否定」 符号,使之与教材一致
\newcommand{\negs}{\sim\!}

\begin{document}
%\\\\\\\\\\\\\\\\\\\\\\\\\\\\\\\\\\\\\\\\\\\\\\\\\\\\\\\\\\\\\\\\\\\\\\\\\\\\\\\

\begin{center}
	\textit{2023·秋·数理逻辑} \qquad \textit{平时作业汇总}  \qquad 大家辛苦啦!  \faCoffee ~ ~\\
	(\textit{英文参考答案} + \textit{作业反馈})
\end{center}


%=================== hw-1 
\noindent\texttt{hw-1 (2023/09/12) }

\emph{p3: 1-(h)} \quad
If $y$ is an integer then $z$ is not real, provided that $x$ is a rational number.


\noindent\textbf{Answer}:

Let
\begin{center}
	\begin{tabular}{l l l }
 $p:$ & $y$ is an integer  \\
 $q:$ & $z$ is a real number  \\
 $r:$ & $x$ is a rational number 
\end{tabular}
\end{center}

\indent 
Therefore we have that $r \to (p \to \neg q)$ or $(r \land p) \to \neg q$.
\hfill $\Box$


%=================== hw-1-feedback
\vspace{1em}
\dotfill hw-1: feedback 
\dotfill

\begin{enumerate}
\item 没有把 “$z$ is {\color{purple} not} real” 中的\textbf{否定联结词}提取出,
进而翻译为公式时缺少否定符号 $\neg$。

\item 对英语的语序产生了错误的判断,将 “{\color{purple} provided that} $x$ is a rational number” 这个句子的成分放置到了错误的地方。

\item 一些同学额外做了教材 \textit{p.3} 第一题中的 $(a)-(h)$, 
但没有注意到 $(c),(e),(g)$ 中的 “{\color{purple} either} ... {\color{purple} or} ...”表达的是
\textbf{不兼容析取},从而对句子产生了不当的翻译。
\end{enumerate}




\vspace{3em}
%\\\\\\\\\\\\\\\\\\\\\\\\\\\\\\\\\\\\\\\\\\\\\\\\\\\\\\\\\\\\\\\\\\\\\\\\\\\\\\\
%\\\\\\\\\\\\\\\\\\\\\\\\\\\\\\\\\\\\\\\\\\\\\\\\\\\\\\\\\\\\\\\\\\\\\\\\\\\\\\\
%=================== hw-2 
\noindent\texttt{hw-2 (2023/09/19)}

\emph{p10}: (7) \quad
Show that the statement form $(((\negs p) \to q)  \to (p \to (\negs q)))$ 
is not a tautology. Find statement forms $\mathscr{A}$ and $\mathscr{B}$ such that 
$(((\negs \mathscr{A}) \to \mathscr{B})  \to (\mathscr{A} \to (\negs \mathscr{B})))$ is a contradiction.

\noindent\textbf{Answer}:

The following truth table shows that $(((\negs p) \to q)  \to (p \to (\negs q)))$ 
is not a tautology.
\begin{center}
	\hspace{8em} 
	\begin{tabular}{@{ }c@{ }@{ }c | c@{ }@{}c@{}@{ }c@{ }@{ }c@{ }@{ }c@{ }@{ }c@{ }@{}c@{}@{ }c@{ }@{}c@{}@{ }c@{ }@{ }c@{ }@{ }c@{ }@{ }c@{ }@{}c@{}@{ }c}
		$p$ & $q$ &  & ( & $\lnot$ & $p$ & $\rightarrow$ & $q$ & ) & $\rightarrow$ & ( & $p$ & $\rightarrow$ & $\lnot$ & $q$ & ) & \\
		\hline 
		T & T &  &  & F & T & T & T &  & \textcolor{red}{F} &  & T & F & F & T &  & \\
		T & F &  &  & F & T & T & F &  & {T} &  & T & T & T & F &  & \\
		F & T &  &  & T & F & T & T &  & {T} &  & F & T & F & T &  & \\
		F & F &  &  & T & F & F & F &  & {T} &  & F & T & T & F &  & \\
	\end{tabular}
\end{center}

When  $\mathscr{A}$ and $\mathscr{B}$  are both tautologies, 
then $(((\negs \mathscr{A}) \to \mathscr{B})  \to (\mathscr{A} \to (\negs \mathscr{B})))$ will be a contradiction. 
For instance, let $\mathscr{A} = \mathscr{B} = (p \to p)$ or 
$\mathscr{A} = \mathscr{B} = (p \lor  \neg p)$.
\hfill $\Box$


\vspace{1em}
除了用真值表这种比较直观的手段外,还有诸多方法。以下答案来自\textit{黄程}同学,经其授权后分享给大家,感谢\textit{黄程}同学\faThumbsUp :

Suppose that $(((\negs p) \to q)  \to (p \to (\negs q)))$  is a tautology. Then the situation that $(\negs p) \to q $ be $T$ and $p \to (\negs q)$ be $F$ will not occur under any valuation. 
But considering $q = T$ and $p = T$,  thus $p \to (\negs q)$ will be $T$. Contradiction! 
Therefore $(((\negs p) \to q)  \to (p \to (\negs q)))$  is not a tautology.

According above answer, when $\mathscr{A}$ and $\mathscr{B}$ be $T$ permanently, then $(((\negs \mathscr{A}) \to \mathscr{B})  \to (\mathscr{A} \to (\negs \mathscr{B})))$ will be a contradiction.
In other words, $\mathscr{A}$ and $\mathscr{B}$ are both tautologies, say, 
$\mathscr{A} = (p \lor  (\negs p))$ and $\mathscr{B} = p \to (q \to p)$.
\hfill $\Box$


%=================== hw-2-feedback
\vspace{1em}
\dotfill hw-2: feedback
\dotfill

\begin{enumerate}
\item 本次作业一共有两问,但存在同学只回答第一问的情况,请大家以后细心。
	
\item 用 $0$ 和 $1$ 来替代 $F$ 和 $T$ 是可以的,有时这样会更为简洁。
	
\item 第一问有同学用一种「简化真值表」来回答,如
\begin{center} 
	\begin{tabular}{@{ }c@{ }@{ }c  c@{ }@{}c@{}@{ }c@{ }@{ }c@{ }@{ }c@{ }@{ }c@{ }@{}c@{}@{ }c@{ }@{}c@{}@{ }c@{ }@{ }c@{ }@{ }c@{ }@{ }c@{ }@{}c@{}@{ }c}
	& ( & $\lnot$ & $p$ & $\rightarrow$ & $q$ & ) & $\rightarrow$ & ( & $p$ & $\rightarrow$ & $\lnot$ & $q$ & ) & \\
	\hline 
	&  & F & T & T & T &  & \textcolor{red}{F} &  & T & F & F & T &  & \\
	&  & F & T & T & F &  & {T} &  & T & T & T & F &  & \\
	&  & T & F & T & T &  & {T} &  & F & T & F & T &  & \\
	&  & T & F & F & F &  & {T} &  & F & T & T & F &  & \\
	\end{tabular}
\end{center}
这是可行且正确的。不过建议还是把 $p$ 和 $q$ 的真值单独列在表前,这样在画真值表找「析取范式」的时候不容易眼花,不过这不是强制性的。
	
\item 第二问要求大家确实为 $\mathscr{A}$ 和 $\mathscr{B}$ 找到某种「命题形式」,很多同学只是声明其为重言式而没有找出具体的「命题形式」,严格来说这是不够的,不过默认大家都掌握了。
\end{enumerate}



\vspace{3em}
%\\\\\\\\\\\\\\\\\\\\\\\\\\\\\\\\\\\\\\\\\\\\\\\\\\\\\\\\\\\\\\\\\\\\\\\\\\\\\\\
%\\\\\\\\\\\\\\\\\\\\\\\\\\\\\\\\\\\\\\\\\\\\\\\\\\\\\\\\\\\\\\\\\\\\\\\\\\\\\\\
%=================== hw-3
\noindent\texttt{hw-3 (2023/09/26)}

\emph{p15}: 11-(a) \qquad
Show, using \textbf{Proposition 1.14} and \textbf{1.17}, that the statement form  
$(  (\neg (p \lor (\neg q)))  \to  (q \to r) )$
is logically equivalent to each of the following.

\hspace{1em} (a) $ ( (\neg (q \to p)) \to ((\neg q) \lor r )) $

\newtcolorbox{mybox}[2][]
{   colback = white, 
	colframe = black, 
	fonttitle = \bfseries,
	colbacktitle = gray, enhanced,
	attach boxed title to top center={yshift=-2mm},
	title=#2,#1}
\begin{mybox}[colback=white,width=19cm,boxrule=0.2mm]{Recall that}
	
	- \textbf{Proposition 1.14}:
	If $\mathscr{B}_1$ is a statement form arising from the statement form $\mathscr{A}$ 
	by substituting the statement form $\mathscr{B}$ for one or more occurrences of the
	statement form $\mathscr{A}$ in $\mathscr{A}_1$, 
	and if $\mathscr{B}$ is logically equivalent to $\mathscr{A}$, 
	then $\mathscr{B}_1$ is logically equivalent to $\mathscr{A}_1$.
	
	-  	\textbf{Proposition 1.17 (De Morgan's Laws)}: 
	Let $\mathscr{A}_1, \mathscr{A}_2, \dotsm \mathscr{A}_n$ be any statement forms. Then:
	\begin{enumerate}
		\item $(\bigvee^n_{i=1} (\neg \mathscr{A}_i))$ is logically equivalent to $( \neg (\bigwedge^n_{i=1} \mathscr{A}_i))$.
		
		\item $(\bigwedge^n_{i=1} ( \neg \mathscr{A}_i))$  is logically equivalent to  $(\neg (\bigvee^n_{i=1} \mathscr{A}_i))$.
	\end{enumerate}
\end{mybox}


\noindent\textbf{Answer}:
Let
$\varphi = (  (\neg (p \lor (\neg q)))  \to  (q \to r) )$ and  
$\chi = ( (\neg (q \to p)) \to ((\neg q) \lor r ))$.

It suffices to show that if $\neg (p \lor (\neg q))$ is logically equivalent to $(\neg (q \to p))$, 
and  $(q \to r)$ is logically equivalent to $ (\neg q) \lor r )$, 
then $\varphi$ is logically equivalent to $\chi$ according to \textbf{Prop. 1.14}.


But it is easy to check, say, using truth table, 
that
\[\begin{array}{r l l}
	\neg (p \lor (\neg q))  &\leftrightarrow&  (\neg (q \to p))  \qquad \text{ and }\\
	
	(q \to r) &\leftrightarrow&  (\neg q) \lor r )   
\end{array}\]
are tautologies, which means that $(\neg (p \lor (\neg q)))$ and $(\neg (q \to p))$, 
$(q \to r)$ and $(\neg q) \lor r ) $ are logically equivalent, respectively. 
\hfill $\Box$


\vspace{1em}
\dotfill  no feedback for hw-3
\dotfill




\vspace{3em}
%\\\\\\\\\\\\\\\\\\\\\\\\\\\\\\\\\\\\\\\\\\\\\\\\\\\\\\\\\\\\\\\\\\\\\\\\\\\\\\\
%\\\\\\\\\\\\\\\\\\\\\\\\\\\\\\\\\\\\\\\\\\\\\\\\\\\\\\\\\\\\\\\\\\\\\\\\\\\\\\\
%=================== hw-4
\noindent\texttt{hw-4 (2023/10/10)}

\emph{p.19}: 13-(a) \quad
Find statement forms in \textbf{conjunctive normal form} which are logically equivalent to the following:
\[
(a) \qquad (((\neg p)  \lor q) \to r)
\] 

\noindent\textbf{Answer}: 
Here we will use \textit{three} methods to find some \textbf{conjunctive normal forms} (CNF) of $(\neg p \lor q) \to r$, the former two are from our \textsf{Textbook}, while the third one is new.

In the first place, let 
\[
\varphi = (\neg p \lor q) \to r.
\]

\noindent\textit{Method-(1)}

First we construct a truth table of $\varphi$'s {\color{red} negation}:
\begin{center}
	\begin{tabular}{@{ }c@{ }@{ }c@{ }@{ }c | c@{ }@{}c@{}@{}c@{}@{ }c@{ }@{ }c@{ }@{ }c@{ }@{ }c@{ }@{}c@{}@{ }c@{ }@{ }c@{ }@{}c@{ }}
		$p$ & $q$ & $r$ & {\color{red} $\lnot$} & ( & ( & $\lnot$ & $p$ & $\lor$ & $q$ & ) & $\rightarrow$ & $r$ & )\\
		\hline 
		1 & 1 & 1 & 0 &  &  & 0 &  & 1 &  &  & 1 &  & \\
		\underline{1} & \underline{1} & \underline{0} & \textcolor{red}{1} &  &  & 0 &  & 1 &  &  & 0 &  & \\
		1 & 0 & 1 & 0 &  &  & 0 &  & 0 &  &  & 1 &  & \\
		1 & 0 & 0 & 0 &  &  & 0 &  & 0 &  &  & 1 &  & \\
		0 & 1 & 1 & 0 &  &  & 1 &  & 1 &  &  & 1 &  & \\
		\underline{0} & \underline{1} & \underline{0} & \textcolor{red}{1} &  &  & 1 &  & 1 &  &  & 0 &  & \\
		0 & 0 & 1 & 0 &  &  & 1 &  & 1 &  &  & 1 &  & \\
		\underline{0} & \underline{0} & \underline{0} & \textcolor{red}{1} &  &  & 1 &  & 1 &  &  & 0 &  & \\
	\end{tabular}
\end{center}

The combinations which give $\neg \varphi$ value $1$ are $110$, $010$ and $000$. 
Thus a \textbf{disjunctive normal form} of $\neg \varphi$ is
\[
\chi = (p \land q \land \neg r) \lor (\neg p \land q \land \neg r) \lor (\neg p \land  \neg q \land \neg r)
\] 
It is clear that $\chi$ is logically equivalent to $\neg \varphi$, 
hence $\neg \chi $ is  logically equivalent to $ \neg \neg \varphi$, i.e., $\varphi$.

Then, by the \textbf{De Morgan’s laws}, we have
\[\begin{array}{l l l}
	\neg \chi 
	&=&   
	\neg [ (p \land q \land \neg r) \lor (\neg p \land q \land \neg r) \lor (\neg p \land  \neg q \land \neg r)  ] \\
	
	&\equiv&   
	{\color{red} \neg} (p \land q \land \neg r) \land {\color{red} \neg} (\neg p \land q \land \neg r) \land {\color{red} \neg} (\neg p \land  \neg q \land \neg r)  \\
	
	&\equiv&   
	( \neg p \lor \neg q \lor  \neg \neg r) \land  ( \neg \neg p \lor \neg q \lor \neg \neg r) \land  (\neg \neg p \lor \neg  \neg q \lor \neg \neg r)  \\
	
	&\equiv&   
	( \neg p \lor \neg q \lor  r) \land  ( p \lor \neg q \lor r) \land  ( p \lor  q \lor r)  \\
\end{array}\]

Therefore, $ ( \neg p \lor \neg q \lor  r) \land  ( p \lor \neg q \lor r) \land  ( p \lor  q \lor r) $ is a CNF of $\varphi$.
\hfill $\Box$

(\textbf{NB}: here we using the symbol expression “$\alpha \equiv \beta$” \; to denote that formula $\alpha$ is logically equivalent to $\beta$)



\vspace{1em}
\noindent\textit{Method-(2)}

\[\begin{array}{l l l l}
	\varphi 
	&=& 
	(\neg p \lor q \to r) \\
	
	&\equiv&
	\neg ( \neg p \lor q) \lor r  & \text{(by \textbf{material implication},}   \text{ cf. p.7: Example 1.4-(a) )}\\
	
	&\equiv&
	(\neg \neg p \land \neg q) \lor r & \text{(by the \textbf{De Morgan’s laws})}  \\
	
	&\equiv&
	( p \land \neg q) \lor r   \\
	
	&\equiv&
	(p \lor r) \land (\neg q \lor r) & (\text{by the \textbf{distribution} of \;} (\lor\text{-}\land), 
	\text{ cf. p.10, Exercises-6-(b)} )\\ 
\end{array}\]

Hence $(p \lor r) \land (\neg q \lor r)$ is a CNF of $\varphi$.
\hfill $\Box$



\vspace{1em}

\noindent\textit{Method-(3)}

Similarly, we construct a truth table for $\varphi$ (notice that, not for the \textit{negation} of $\varphi$):

\begin{center}
	\begin{tabular}{@{ }c@{ }@{ }c@{ }@{ }c | c@{ }@{}c@{}@{ }c@{ }@{ }c@{ }@{ }c@{ }@{ }c@{ }@{}c@{}@{ }c@{ }@{ }c@{ }@{ }c}
		$p$ & $q$ & $r$ &  & ( & $\lnot$ & $p$ & $\lor$ & $q$ & ) & $\rightarrow$ & $r$ & \\
		\hline 
		1 & 1 & 1 &  &  & 0 & 1 & 1 & 1 &  &                  1 & 1 & \\
		\underline{1} & \underline{1} & \underline{0} &  &  & 0 & 1 & 1 & 1 &  & \textcolor{red}{0} & 0 & \\
		1 & 0 & 1 &  &  & 0 & 1 & 0 & 0 &  &                 1  & 1 & \\
		1 & 0 & 0 &  &  & 0 & 1 & 0 & 0 &  & 				 1 & 0 & \\
		0 & 1 & 1 &  &  & 1 & 0 & 1 & 1 &  & 				 1 & 1 & \\
		\underline{0} & \underline{1} & \underline{0} &  &  & 1 & 0 & 1 & 1 &  & \textcolor{red}{0} & 0 & \\
		0 & 0 & 1 &  &  & 1 & 0 & 1 & 0 &  & 				 1 & 1 & \\
		\underline{0} & \underline{0} & \underline{0} &  &  & 1 & 0 & 1 & 0 &  & \textcolor{red}{0} & 0 & \\
	\end{tabular}
\end{center}

The combinations which give $\varphi$ value $0$ are $110$, $010$ and $000$. 
Then according to these truth combinations, we can construct three \textbf{disjunctive formulas} as follows,
\[\begin{array}{l l l}
	\varphi_1 
	&=&
	(\neg p \lor \neg q \lor r) \\ 
	
	\varphi_2
	&=&
	( p \lor \neg q \lor r) \\
	
	\varphi_3
	&=&
	( p \lor  q \lor r)  \\
\end{array}\]

Next, we connect above three formulas in a conjunctive form, that is,
\[
\varphi_1 \land \varphi_2 \land \varphi_3 = ( \neg p \lor \neg q \lor  r) \land  ( p \lor \neg q \lor r) \land  ( p \lor  q \lor r)
\]

It is easy to check that  $\varphi_1 \land \varphi_2 \land \varphi_3$ is a CNF of $\varphi$. 
And as we can see, the result in current \textit{Method-(3)} is same as the  \textit{Method-(1)}. 
\hfill $\Box$



%%%%%%%%%%%%%%%%%%%%%%%%%%%%%%%%%%%%%%%%%%%%%%%%%%%%%%%%%%%%%%%%%%%%%%%%%%%%%%%%
%\vspace{2em}  % 如果用 latex 答题 请把这个垂直间距删除,或调整为自己喜欢的间距
\vspace{3em}

\emph{p.26}: 21 \quad
Suppose that $\mathscr{A}_1,\mathscr{A}_2,\dots,\mathscr{A}_n$;  $\therefore\mathscr{A}$ is a valid argument form. Prove that $\mathscr{A}_1,\mathscr{A}_2,\dots,\mathscr{A}_{n-1}$;  $\therefore (\mathscr{A}_n \to \mathscr{A})$ is also a valid argument form.

\noindent\textbf{Proof}:

First, suppose that $\mathscr{A}_1,\mathscr{A}_2,\dots,\mathscr{A}_n$;  $\therefore\mathscr{A}$ is a valid argument form, but $\mathscr{A}_1,\mathscr{A}_2,\dots,\mathscr{A}_{n-1}$;  $\therefore (\mathscr{A}_n \to \mathscr{A})$ is not. 

Then there is an assignment of truth values to the statement variables such that 
$\mathscr{A}_1,\mathscr{A}_2,\dots,\mathscr{A}_{n-1}$ takes value $T$ 
while $(\mathscr{A}_n \to \mathscr{A})$ takes value $F$, that is, 
$\mathscr{A}_n $ is $T$ but $\mathscr{A}$ takes $F$. 
However, this contradicts to our assumption that $\mathscr{A}_1,\mathscr{A}_2,\dots,\mathscr{A}_n$;  $\therefore\mathscr{A}$ is a valid argument form.
\hfill $\Box$


%=================== hw-4: feedback
\vspace{1em}
\dotfill
hw-4: feedback
\dotfill

\begin{enumerate}
	\item 还是有同学少写题目呀,题目少写的话想给你们找分都很难了。考试的时候也差不多,尽量不要空题不做呀 \faSadTear[regular]
	
	\item 还有很多同学写证明的时候,一句话中往往不写「定语」和「状语」,比如会出现如下情况:
	\begin{center}
		\textit{所以 $\varphi$ .... } 
	\end{center}
	所以$\varphi$什么呢?$\varphi$是重言式?$\varphi$是矛盾式?这些都是需要额外加以说明的。
\end{enumerate}






\vspace{4em}
%\\\\\\\\\\\\\\\\\\\\\\\\\\\\\\\\\\\\\\\\\\\\\\\\\\\\\\\\\\\\\\\\\\\\\\\\\\\\\\\
%\\\\\\\\\\\\\\\\\\\\\\\\\\\\\\\\\\\\\\\\\\\\\\\\\\\\\\\\\\\\\\\\\\\\\\\\\\\\\\\
%=================== hw-5
\noindent\texttt{hw-5 (2023/10/17)}

\emph{p.36}: 1-(c) \quad
Write out proofs in $L$ for the following $wfs$.
\[
(c) \qquad  (p_1 \to (p_1 \to p_2)) \to (p_1 \to p_2)
\] 


\noindent\textbf{Proof}: 

\noindent\textit{Method-(1)}

\begin{enumerate}
	\item $ (p_1 \to (p_1 \to p_2)) \to ((p_1 \to p_1) \to (p_1 \to p_2)) $
	\hfill (instance of $L2$)
	
	\item $ [(p_1 \to (p_1 \to p_2)) \to ((p_1 \to p_1) \to (p_1 \to p_2))] \to  $
	
	$
	[( (p_1 \to (p_1 \to p_2)) \to (p_1 \to p_1))  \to
	( (p_1 \to (p_1 \to p_2)) \to (p_1 \to p_2)) ]
	$
	\hfill (instance of $L2$)
	
	\item $ ( (p_1 \to (p_1 \to p_2)) \to (p_1 \to p_1))  \to
	( (p_1 \to (p_1 \to p_2)) \to (p_1 \to p_2))$
	\hfill ($1+2,MP$)
	
	\item $p_1 \to ((p_1 \to p_2) \to p_1)$ 
	\hfill (instance of $L1$)
	
	\item $[p_1 \to ((p_1 \to p_2) \to p_1)] \to 
	[ (p_1 \to (p_1 \to p_2))  \to (p_1 \to p_1) ]$ 
	\hfill (instance of $L2$)
	
	\item $(p_1 \to (p_1 \to p_2))  \to (p_1 \to p_1)$
	\hfill ($4+5, MP$)
	
	\item $(p_1 \to (p_1 \to p_2)) \to (p_1 \to p_2)$ 
	\hfill ($3+6, MP$)
\end{enumerate}
The proof for (c) is not unique, of course.
\hfill $\Box$


%---------------------------------------
\vspace{1em} 

\noindent\textit{Method-(2)}

\begin{enumerate}
	\item $p_1 \to ((p_1 \to p_1) \to p_1)$  
	\hfill (instance of $L1$)
	
	\item $(p_1 \to ((p_1 \to p_1) \to p_1))   \to   (  (p_1 \to (p_1 \to p_1)) \to (p_1 \to p_1) )$ 
	\hfill (instance of $L2$)
	
	\item $(p_1 \to (p_1 \to p_1)) \to (p_1 \to p_1)$ 
	\hfill ($1+2, MP$)
	
	\item $p_1 \to (p_1 \to p_1)$ 
	\hfill (instance of $L1$)
	
	\item $(p_1 \to p_1)$
	\hfill ($3+4, MP$)
	
	\item $(p_1 \to p_1) \to ( (p_1 \to (p_1 \to p_2)) \to (p_1 \to p_1) )$  
	\hfill (instance of $L1$)
	
	\item $(p_1 \to (p_1 \to p_2)) \to (p_1 \to p_1)$ 
	\hfill ($5+6, MP$)
	
	\item $( p_1 \to (p_1 \to p_2))  \to  ( (p_1 \to p_1)  \to (p_1 \to p_2) )$  
	\hfill (instance of $L2$)
	
	\item $ [(p_1 \to (p_1 \to p_2)) \to ((p_1 \to p_1) \to (p_1 \to p_2))] \to  $
	
	$
	[( (p_1 \to (p_1 \to p_2)) \to (p_1 \to p_1))  \to
	( (p_1 \to (p_1 \to p_2)) \to (p_1 \to p_2)) ]
	$
	\hfill (instance of $L2$)
	
	\item $( (p_1 \to (p_1 \to p_2)) \to (p_1 \to p_1))  \to
	( (p_1 \to (p_1 \to p_2)) \to (p_1 \to p_2)) $  
	\hfill ($8+9,MP$)
	
	\item $(p_1 \to (p_1 \to p_2)) \to (p_1 \to p_2)$ 
	\hfill ($7+10, MP$)
\end{enumerate}


%---------------------------------------
\vspace{1em}

\noindent\textit{Method-(3)}

\begin{enumerate}
	\item  $ \{ (p_1 \to p_2) \to [ ( (p_1 \to p_2) \to (p_1 \to p_2) )  \to (p_1 \to p_2)   ] \} \to $
	
	$ \{ [ (p_1 \to p_2) \to ( (p_1 \to p_2) \to (p_1 \to p_2) ) ] \to [(p_1 \to p_2) \to (p_1 \to p_2)]  \}$
	\hfill (instance of $L2$)
	
	
	\item $(p_1 \to p_2) \to [ ( (p_1 \to p_2) \to (p_1 \to p_2) )  \to (p_1 \to p_2)   ]$
	\hfill (instance of $L1$)
	
	\item $[ (p_1 \to p_2) \to ( (p_1 \to p_2) \to (p_1 \to p_2) ) ] \to [(p_1 \to p_2) \to (p_1 \to p_2)] $ 
	\hfill ($1+2, MP$)
	
	\item $(p_1 \to p_2) \to ( (p_1 \to p_2) \to (p_1 \to p_2) ) $ 
	\hfill (instance of $L1$)
	
	\item $(p_1 \to p_2) \to (p_1 \to p_2)$  
	\hfill ($3+4, MP$)
	
	\item $[ (p_1 \to p_2) \to (p_1 \to p_2)]   \to [ ((p_1 \to p_2) \to p_1)  \to ((p_1 \to p_2) \to p_2)  ]$ 
	\hfill (instance of $L2$)
	
	\item $((p_1 \to p_2) \to p_1)  \to ((p_1 \to p_2) \to p_2)$ 
	\hfill ($5+6, MP$)
	
	\item $ [ ((p_1 \to p_2) \to p_1)  \to ((p_1 \to p_2) \to p_2) ]  \to $ 
	
	$
	[ p_1 \to ( ((p_1 \to p_2) \to p_1)  \to ((p_1 \to p_2) \to p_2) )  ]$
	\hfill (instance of $L1$)
	
	\item $p_1 \to ( ((p_1 \to p_2) \to p_1)  \to ((p_1 \to p_2) \to p_2) )  $ 
	\hfill ($7+8, MP$)
	
	\item $ [p_1 \to ( ((p_1 \to p_2) \to p_1)  \to ((p_1 \to p_2) \to p_2) )   ] \to$
	
	$[ (p_1 \to ((p_1 \to p_2) \to p_1))   \to (p_1 \to ((p_1 \to p_2) \to p_2))   ]$
	\hfill (instance of $L2$)
	
	\item $(p_1 \to ((p_1 \to p_2) \to p_1))   \to (p_1 \to ((p_1 \to p_2) \to p_2))  $  
	\hfill ($9+10,MP$)
	
	\item $p_1 \to ((p_1 \to p_2) \to p_1)$  
	\hfill (instance of $L1$)
	
	\item $ p_1 \to ((p_1 \to p_2) \to p_2)$  
	\hfill ($11+12, MP$)
	
	\item $[ p_1 \to ((p_1 \to p_2) \to p_2)]  \to [ (p_1 \to (p_1 \to p_2)) \to (p_1 \to p_2) ]$
	\hfill (instance of $L2$) 
	
	\item  $(p_1 \to (p_1 \to p_2)) \to (p_1 \to p_2)$
	\hfill ($13+14, MP$) 
\end{enumerate}


\vspace{1em}

(ps. 上面公式中的 \textit{中括号} $[ \;]$ 和 \textit{花括号} $\{ \; \}$ 是起辅助作用的,为的是方便大家观看。但应注意的是,其本身不是命题逻辑公理系统$L$中的符号 !!!)

\vspace{2em}

\emph{p.37}: 5 \quad
The rule $HS$ is an example of a legitimate additional rule of deduction for $L$. Is the following rule legitimate in the same sense: from the $wfs.$ $\mathscr{B}$ and $(\mathscr{A} \to (\mathscr{B} \to \mathscr{C}))$, deduce $(\mathscr{A} \to \mathscr{C})$ ?

\noindent\textbf{Answer}:   

\noindent\textit{Method-(1)} (without using the \textbf{Deduction Theorem})

\begin{enumerate}
	\item $\mathscr{B}$
	\hfill (assumption)
	
	\item $(\mathscr{A} \to (\mathscr{B} \to \mathscr{C}))$
	\hfill (assumption)
	
	
	\item $(\mathscr{A} \to (\mathscr{B} \to \mathscr{C})) \to ((\mathscr{A} \to \mathscr{B}) \to (\mathscr{A} \to \mathscr{C})) $
	\hfill ($L2$)
	
	
	\item $( (\mathscr{A} \to \mathscr{B}) \to (\mathscr{A} \to \mathscr{C}) )$
	\hfill ($2 + 3, MP$)
	
	\item $(\mathscr{B} \to (\mathscr{A} \to \mathscr{B}))$
	\hfill ($L1$)
	
	\item $(\mathscr{A} \to \mathscr{B})$
	\hfill ($1+5, MP$)
	
	\item $(\mathscr{A} \to \mathscr{C})$
	\hfill ($6+4, MP$)
\end{enumerate}

Hence this rule is a legitimate additional rule of deduction for $L$.
\hfill $\Box$



%---------------------------------------------------------
\vspace{1em}
\noindent\textit{Method-(2)} (using the \textbf{Deduction Theorem})

We first show that 
\[
\{\mathscr{B}, (\mathscr{A} \to (\mathscr{B} \to \mathscr{C}))\} \cup \{ \mathscr{A}\} \vdash_L \mathscr{C}.
\]

We write out a deduction for above one as follows:

\begin{enumerate}
	\item $\mathscr{B}$  
	\hfill (assumption)
	
	\item $(\mathscr{A} \to (\mathscr{B} \to \mathscr{C}))$
	\hfill (assumption)
	
	\item $\mathscr{A}$
	\hfill (assumption)
	
	\item $(\mathscr{B} \to \mathscr{C})$
	\hfill ($2+3,MP$)
	
	\item $\mathscr{C}$
	\hfill ($1+4, MP$)
\end{enumerate} 

Hence by the \textbf{Deduction Theorem}, we have
\[
\{\mathscr{B}, (\mathscr{A} \to (\mathscr{B} \to \mathscr{C}))\} \vdash_L \mathscr{A} \to \mathscr{C}.
\]
as required.
\hfill $\Box$

\vspace{1em}

\dotfill hw-5: feedback
\dotfill

\begin{itemize}
	\item  很多同学都误解了什么是一个「$L$中的证明」,在其中,是不能出现 “\textit{假设}”、“\textit{因为-所以}”这样的字眼的。因此
	\textit{p.36} 1-(c) 的证明也不能用「演绎定理」,这个是内定理证明,证明的序列中出现的只能是\textit{公理} 或者由前面的公式使用 $MP$ 得到。  还请大家特别要注意这点!
\end{itemize}






\vspace{4em}
%\\\\\\\\\\\\\\\\\\\\\\\\\\\\\\\\\\\\\\\\\\\\\\\\\\\\\\\\\\\\\\\\\\\\\\\\\\\\\\\
%\\\\\\\\\\\\\\\\\\\\\\\\\\\\\\\\\\\\\\\\\\\\\\\\\\\\\\\\\\\\\\\\\\\\\\\\\\\\\\\
%=================== hw-6
\noindent\texttt{hw-6 (2023/10/31)}  \textit{期中作业}

\emph{p.44}: (8) \quad
Let $\mathscr{A}$ be a \textit{wf.} $((\neg p_1 \to p_2) \to (p_1 \to \neg p_2))$. 
Show that $L^+$, obtained by including this $\mathscr{A}$ as a new axiom, has a larger set of theorems than $L$. Is $L^+$ a consistent extension of $L$?
(\textit{注意:此题有两问})

\noindent\textbf{Proof}:

\begin{center}
	\begin{tabular}{@{ }c@{ }@{ }c | c@{ }@{}c@{}@{ }c@{ }@{ }c@{ }@{ }c@{ }@{ }c@{ }@{}c@{}@{ }c@{ }@{}c@{}@{ }c@{ }@{ }c@{ }@{ }c@{ }@{ }c@{ }@{}c@{}@{ }c}
		$p_1$ & $p_2$ &  & ( & $\lnot$ & $p_1$ & $\to$ & $p_2$ & ) & $\to$ & ( & $p_1$ & $\to$ & $\lnot$ & $p_2$ & ) & \\
		\hline 
		T & T &  &  & F & T & T & T &  & \textcolor{purple}{F} &  & T & F & F & T &  & \\
		T & F &  &  & F & T & T & F &  & \textcolor{purple}{T} &  & T & T & T & F &  & \\
		F & T &  &  & T & F & T & T &  & \textcolor{purple}{T} &  & F & T & F & T &  & \\
		F & F &  &  & T & F & F & F &  & \textcolor{purple}{T} &  & F & T & T & F &  & \\
	\end{tabular}
\end{center}

\textit{For the first question}:
Obviously $\mathscr{A} = ((\neg p_1 \to p_2) \to (p_1 \to \neg p_2))$ is not a tautology by above truth table, 
then by the \textbf{Soundness Theorem}, 
$\mathscr{A}$ is {\color{purple} not} a theorem of $L$, 
while it is a theorem of $L^+$, 
therefore $L^+$ has a larger set of theorems than $L$.

\textit{For the second question}:
$L^+$ is consistent. 
For suppose otherwise, then there is a formula $\mathscr{B}$ such that 
$\vdash_{L^+} \mathscr{B}$  and	$\vdash_{L^+} \neg \mathscr{B}$. 
Since $L^+$ is obtained by including $\mathscr{A} = ((\neg p_1 \to p_2) \to (p_1 \to \neg p_2))$ as an extra axiom then $L$, 
hence we have that (note that the {\color{purple} subscript} of $\vdash$)
\[\begin{array}{l l l}
	\mathscr{A} \vdash_{L} \mathscr{B}  \quad \text{and} \quad	
	\mathscr{A} \vdash_{L} \neg \mathscr{B}. \\
\end{array}\]
By the \textbf{Deduction Theorem}, 
\[
\vdash_{L} \mathscr{A} \to \mathscr{B}  \quad \text{and} \quad	
\vdash_{L} \mathscr{A} \to \neg \mathscr{B}, 
\]
which means that
($\mathscr{A} \to \mathscr{B})$ and $(\mathscr{A} \to \neg \mathscr{B})$ are tautologies according to the \textbf{Soundness Theorem}.
Then by the definition, for any valuation $v$ we have that 
$v(\mathscr{A} \to \mathscr{B})  = T$ and  $v(\mathscr{A} \to \neg \mathscr{B})  = T$,
which implies that $v(\mathscr{A}) = F$, that is, $\mathscr{A}$ is a \textit{contradiction}. But this is impossible by the truth table of $\mathscr{A}$. Contradiction!
\hfill $\Box$



%%%%%%%%%%%%%%%%%%%%%%%%%%%%%%%%%%%%%%%%%%%%%%%%%%%%%%%%%%%%%%%%%%%%%%%%%%%%%%%%
\vspace{2em}  
\noindent\emph{p.44}: (10) \quad
Let $L^{++}$ be the extension of $L$ obtained by including as a fourth axiom \textit{scheme}:
\[
( (\neg \mathscr{A} \to \mathscr{B}) \to (\mathscr{A} \to \neg \mathscr{B})).
\]
Show that $L^{++}$ is inconsistent. (Hint: see Chapter 1 exercise 7 (p.10)) 


\noindent\textbf{Proof}:

\noindent\textit{Method-(1)}

Let 
$\top = (p \to p)$ and $\varphi = (\neg \top \to \top)  \to (\top \to \neg \top)$,
clearly $ \vdash_{L^{++}} \varphi$ (i.e., let $\mathscr{A} = \mathscr{B} = \top$). 
It is easy to check , 
say using truth table, that $\varphi$ is a contradiction, 
hence $\neg \varphi$ is a tautology. 
By the \textbf{Completeness Theorem}, $\vdash_L \neg \varphi$, 
and thus $\vdash_{L^{++}} \neg \varphi$ since $L^{++}$ is a extension of $L$.

But we have that $\vdash_{L^{++}} \varphi$ and $\vdash_{L^{++}} \neg \varphi$, by the definition, $L^{++}$ is inconsistent as required.
\hfill $\Box$


\noindent\textit{Method-(2)}

(下面这个证明来自 \textit{吴家儒}\ 同学,这种证明很直接且颇具暴力美学,
再次感谢\textit{家儒}同学为我们带来如此精彩的证明 \faHeart[regular] \faHeart[regular] \faHeart[regular])

Since $\vdash_L (p \to p)$ (cf. \textit{Example 2.7-(a)} in page 31),  we have that
$
\vdash_{L^{++}} (p \to p)
$ obviously.
And let $(L4)$ denotes the  \textit{fourth axiom scheme} of $L^{++}$, that is,
\[
(L4) \qquad ( (\neg \mathscr{A} \to \mathscr{B}) \to (\mathscr{A} \to \neg \mathscr{B})).
\]


Considering the  following proof sequence  in $L^{++}$:
\begin{enumerate}
	\item $[\neg (p \to p) \to (p \to p)]   \to [(p \to p) \to \neg (p \to p)]$ 
	\hfill  (instance of $L4$)
	
	\item $[ (\neg (p \to p) \to (p \to p))   \to ((p \to p) \to \neg (p \to p)) ]$ $\to$
	
	$[
	((\neg (p \to p) \to (p \to p)) \to (p \to p) )  \to 
	(
	(\neg (p \to p) \to (p \to p)) \to \neg (p \to p)
	)
	]$
	\hfill  (instance of $L2$)
	
	\item $((\neg (p \to p) \to (p \to p)) \to (p \to p) )  \to 
	(
	(\neg (p \to p) \to (p \to p)) \to \neg (p \to p)
	)$
	\hfill ($1 + 2, MP$)
	
	\item $(p \to p) \to [(\neg (p \to p) \to (p \to p)) \to (p \to p)]$ 
	\hfill (instance of $L1$)
	
	\item $(p \to p) $ 
	\hfill ($(p\to p)$ is a theorem of $L$, so is for $L^{++}$)
	
	\item $(\neg (p \to p) \to (p \to p)) \to (p \to p) $
	\hfill ($4+5, MP$)
	
	\item $(\neg (p \to p) \to (p \to p)) \to \neg (p \to p)$
	\hfill ($6+3, MP$)
	
	\item $(p \to p) \to ( \neg (p \to p) \to (p \to p))$ 
	\hfill (instance of $L1$)
	
	\item $ \neg (p \to p) \to (p \to p)$ 
	\hfill ($5+8, MP$)
	
	\item $\neg (p \to p)$
	\hfill ($9+7, MP$)
\end{enumerate}

\noindent 
Hence $\vdash_{L^{++}} \neg (p \to p)$, together with previous  $\vdash_{L^{++}} (p \to p)$, $L^{++}$ is inconsistent as desired.
\hfill $\Box$



\vspace{1em}
\dotfill hw-6: feedback
\dotfill

\begin{enumerate}
	\item 大部分人还是没有区分「元语言」和「对象语言」,所以严格来说很多人的回答都是不合法的甚至是错误的。不过改作业的时候已经采取十分宽容的态度了,还希望大家一定要重视这点,这对后续的逻辑学习是十分重要的。
	
	\item 依旧强烈建议{\color{purple} 不要}使用「简化真值表」,这并不是说「简化真值表」是什么洪水猛兽大家碰不得,只不过照现在的作业来看,一画「简化真值表」就容易画错。
	
	\item 虽然很多同学借鉴了教材 \textit{p.205} 的提示,但这种提示往往省略了超多细节,
	这些细节应该要补充完整的,直接抄书行不得!一个证明首先要说服自己才能说服别人!
	
	\item 建议用\textbf{黑笔}!\textbf{黑笔}!\textbf{黑笔}!作答,期末考试时也是一样的。
	
	\item 很多同学都误用了 $(L3): (\neg \mathscr{A} \to \neg \mathscr{B}) \to (\mathscr{B} \to \mathscr{A})$ 公理,如下的公式并{\color{purple} 不是} $(L3)$公理的一个实例:
	\[
	(p \to q) \to (\neg q \to \neg p)  \qquad \text{或} \qquad
	(p \to \neg q) \to (q \to \neg p)
	\]
	单单只使用公理模式 $(L3)$ 得不到 上述公式是 $L$ 的定理的,注意否定符号的位置。
	
	
	\item 同样容易误用的是 \textbf{Proposition 2.19}:
	\begin{quotation}
		Let $L^\ast$ be a consistent extension of $L$ and let $\varphi$ be a formula which is not a theorem of $L^\ast$. Then $L^{\ast\ast}$ is also consistent, where $L^{\ast\ast}$ is the extension of $L$ obtained from $L^\ast$ by including $(\neg \varphi)$ as an additional axiom. (p. 40)
	\end{quotation}
	显然$L$是其本身的一个一致扩张,并且很多人做第8题第二问的时候,确实证明了 $\not \vdash_L \neg \mathscr{A}$,然后直接运用 \textbf{Prop. 2.19} 就说 $L^+$ 是 $L$的一致扩张,这中间其实还有一个 gap 要补充的。
	
	根据 \textbf{Prop. 2.19} 和 $\not \vdash_L \neg \mathscr{A}$ 我们只能得到 $L \cup \{ \neg \neg \mathscr{A}\}$ 是 一致的(注意否定的个数),而题目中的是 $L^+ = L \cup \{ \mathscr{A}\}$。虽然语义直观上 $\mathscr{A}$ 和 $\neg \neg \mathscr{A}$ 是一个意思,但是仅仅作为字符串来说二者是完全不同的东西。因此,如果硬是要用 \textbf{Prop. 2.19} 的话,我们就必须还得论证:$L \cup \{ \neg \neg \mathscr{A}\}$ 和 $  L \cup \{ \mathscr{A} \}$ 是同一个系统。然而这在教材中是没有明确说明的。
	
	\item 抄作业的情况有点严重呀!虽然鼓励同学们相互讨论,但写作业的时候也别直接抄呀,都做对就还好啦,错都错一样的话就很难说过去了 \textit{:(} 
\end{enumerate}



\vspace{4em}
%\\\\\\\\\\\\\\\\\\\\\\\\\\\\\\\\\\\\\\\\\\\\\\\\\\\\\\\\\\\\\\\\\\\\\\\\\\\\\\\
%\\\\\\\\\\\\\\\\\\\\\\\\\\\\\\\\\\\\\\\\\\\\\\\\\\\\\\\\\\\\\\\\\\\\\\\\\\\\\\\
%=================== hw-7
\noindent\texttt{hw-7 (2023/11/07)}

\emph{p.49}: 2-(c) \quad
Translate each of the following statements into symbols, {\color{purple} first} using no existential quantifiers, and {\color{purple} second} using no universal quantifiers. 

\qquad (c) \hspace{3cm}  \textit{No mouse is heavier than any elephant.}

(\textit{注意:题目要求大家要分别用“全称量词”和“存在量词”符号化句子,因此你的翻译至少有两句})

\noindent\textbf{Answer}:   

Let 
\[\begin{array}{l l l}
	M(x):& x \text{ is a \textit{mouse}}\\
	E(x): & x \text{ is an \textit{elephant}}\\
	H(x,y): & x \text{ is \textit{heavier than}\;} y \\
\end{array}\]

Using \textit{no} existential quantifier: 
\begin{enumerate}
	\item $(\forall x) (\forall y) (M(x) \land E(y) \to \neg H(x,y))$, or
	
	\item $(\forall x) (\forall y) (M(x) \to (E(y) \to \neg H(x,y)))$, or
	
	\item $(\forall x)  (M(x) \to (\forall y) (E(y) \to \neg H(x,y)))$, or
	
	\item any other reasonable answers.
\end{enumerate}

\vspace{1em}

Using \textit{no} universal quantifier:
\begin{enumerate}
	\item $\neg (\exists x) (\exists y) (M(x) \land E(y) \land H(x,y))$, or
	
	\item $\neg (\exists x)  (M(x) \land (\exists y) (E(y) \land H(x,y)))$, or
	
	\item any other sensible answers. 
	\hfill $\Box$
\end{enumerate}



\vspace{1em}

\dotfill hw-7: feedback
\dotfill

这次作业大部分人都写得很好,但有两点还请大家要尤其注意下:

\begin{enumerate}
	\item 对谓词的拆解不完全。有人的用诸如 $D(x)$ 这样的符号来表示谓词\; “ $x$ \textit{比老鼠重}”,便会有如下的翻译($E(x)$表示\; “$x$\textit{是大象}”):
	\[
	(\forall x) (E(x) \to D(x))
	\]
	这种翻译就没有把谓词\; “\textit{$\dots$ 比  $\dots$重}” \;符号化。
	
	\item 如果用 $H(x,y)$ 表示 \; “\textit{$x$ 比  $y$ 重}” ,有些同学会把 $H(y,x)$ 理解为 $H(x,y)$ 的否定,即认同 $H(y,x) = \neg H (x,y)$,进而有如下的翻译:
	\[
	(\forall x) (\forall y) (M(x) \land E(y) \to  H(y,x))  \qquad (\divideontimes)
	\]
	这种翻译直观上好像可以,但仔细想想,如果我们令 $x =y$,就会产生下面的问题
	\[
	H(x,x) \qquad = \qquad \neg H(x,x)
	\]
	
	采用这种翻译的同学其实在脑海中预设了 $H(x,y)$ 是一个\textit{严格偏序关系}(即“反自反+传递”),但这就需要{\color{purple} 额外的}一阶公式来说明$H$是一个严格偏序关系,因此严格来说上面的翻译$(\divideontimes)$是不符合题意的。不过改作业还是采取了宽容的态度,但这并不说明这种答案可行,请特别注意这点!
\end{enumerate}




\vspace{4em}
%\\\\\\\\\\\\\\\\\\\\\\\\\\\\\\\\\\\\\\\\\\\\\\\\\\\\\\\\\\\\\\\\\\\\\\\\\\\\\\\
%\\\\\\\\\\\\\\\\\\\\\\\\\\\\\\\\\\\\\\\\\\\\\\\\\\\\\\\\\\\\\\\\\\\\\\\\\\\\\\\
%=================== hw-8
\noindent\texttt{hw-8 (2023/11/14)}

\emph{p.56}: 9-(d) \quad
In each case below, let $\mathscr{A}(x_1)$ be the given \emph{wf}., and let $t$ be the term $f^2_1(x_1,x_3)$. Write out the $wf.$ $\mathscr{A}(t)$ and hence decide in each case whether $t$ is {\color{purple} free for $x_1$} in the given \emph{wf}.
\[
(d) \qquad (\forall x_2) A^3_1 (x_1, f^1_1(x_1),x_2)  \to (\forall x_3)A^1_1(f^2_1(x_1,x_3)).
\]

\newtcolorbox{mybox2}[2][]
{   colback = white, 
	colframe = black, 
	fonttitle = \bfseries,
	colbacktitle = gray, enhanced,
	attach boxed title to top center={yshift=-2mm},
	title=#2,#1}
\begin{mybox2}[colback=white,width=19cm,boxrule=0.2mm]{Recall that}
	
	- {\color{red} $\mathscr{A}(t)$}: if $x_i$ does occur free in $\mathscr{A}(x_1)$, then $\mathscr{A}(t)$ denotes the result of substituting term $t$ for  {\color{purple}every free occurrence} of $x_i$. (cf. \emph{p.54})	
	
	-  $t$ is {\color{red} free} for $x$ in a \emph{wf.} $\phi$: 
	
	\textbf{定义3.11*. (Revised defintion)} \quad
	当一个项 $t$ 可以替换 $\mathscr{A}$ 中变元 $x_i$ 的{\color{purple} 所有自由出现},且不会使得 $t$ 中任何变元与 $\mathscr{A}$ 的其他部分相互作用,我们就称 { \color{red} $t$ 对 $\mathscr{A}$ 中 $x_i$ 是自由的}。
\end{mybox2}


(\textit{注意此题有两问: 你需要 $1)$ 写出$\mathscr{A}(t)$,且 $2)$ 回答$t$在$\mathscr{A}(x_1)$中是否对$x_1$自由})

\noindent\textbf{Answer}:  

Note that in
\[
(d) \qquad (\forall x_2) A^3_1 ({\color{purple} x_1}, f^1_1({\color{purple} x_1}),x_2)  \to (\forall x_3)A^1_1(f^2_1({\color{purple} x_1},x_3)).
\]
$x_1$ has \textit{three} occurrences are free,
hence 
\[
\mathscr{A}(t) = (\forall x_2) A^3_1 ({\color{purple} f^2_1(x_1,x_3)}, f^1_1({\color{purple} f^2_1(x_1,x_3)}),x_2)  \to (\forall x_3)A^1_1(f^2_1({\color{purple} f^2_1(x_1,x_3)},x_3))  
\]
And $t$ is \textit{not} free for $x_1$ in $(d)$ of course. 
\hfill $\Box$


\vspace{1em}
\dotfill hw-8: feedback
\dotfill
\begin{enumerate}
	\item \textit{关于代入后的结果}。对$x_1$的自由出现代入$t$后,一定得在所得的公式中把$t$展开了,仅仅写成
	\[
	(\forall x_2) A^3_1 ({\color{purple} t}, f^1_1({\color{purple} t}),x_2)  \to (\forall x_3)A^1_1(f^2_1({\color{purple} t},x_3))
	\]
	这个样子是不可行滴,且就定义而言,上面这个符号串也不是一个\textit{合式公式}(因为一阶语言的字母表中并没有$t$这样的符号,$t$只是\textit{元语言}中的符号)。
	
	\item \textit{关于符号的写法}。对于全称量词或存在量词,可以采取书上的写法,即 $\forall x_i$ 和 $\exists x_i$ 外面有对括号:
	$ (\forall x_i) \varphi$ 、 $(\exists x_i) \varphi$。
	比较现代的记法一般省略会这对括号,直接写作: $\forall x_i \varphi, \exists x_i \varphi $。但有些同学会在把变元用括号括起来,从而有形如
	\[
	\forall (x_i) \varphi \qquad \exists (x_i) \varphi
	\]
	这样的写法。不过这种写法既不太美观也不通用,有时还会让人看得比较困惑,所以还是建议不要自创记法为好。
	
	\item \textit{关于代入自由}。一个项$t$对于某个公式$\varphi$中的变元$x$是自由的,一定是相对于整个公式$\varphi$来说的,当$\varphi$是一个蕴含式(或者其他复合公式)时,没有「$t$对$\varphi$的前件代入自由」或者「$t$对$\varphi$的后件不是代入自由」这类说法。
\end{enumerate}



\vspace{4em}
%\\\\\\\\\\\\\\\\\\\\\\\\\\\\\\\\\\\\\\\\\\\\\\\\\\\\\\\\\\\\\\\\\\\\\\\\\\\\\\\
%\\\\\\\\\\\\\\\\\\\\\\\\\\\\\\\\\\\\\\\\\\\\\\\\\\\\\\\\\\\\\\\\\\\\\\\\\\\\\\\
%=================== hw-9
\noindent\texttt{hw-9 (2023/11/21)}

\emph{p.59}: 11 \quad
Let $\mathscr{L}$ be the first order language which includes (besides variables, punctuation, 
connectives and quantifier) the individual constant $a_1$, the function letter $f^2_1$ and the predicate letter $A^2_2$. 
Let $\mathscr{A}$ denote the \textit{wf}.
\[
(\forall x_1)(\forall x_2)( A^2_2 ( f^2_1(x_1,x_2), a_1)  \to A^2_2 (x_1,x_2) ).
\]
Define an interpretation $I$ of $\mathscr{A}$ as follows. 
$D_I$ is $\mathbb{Z}$, $\bar{a}_1$ is $0$, $\bar{f^2_1} (x, y)$ is $x-y$, 
$\bar{A^2_2}(x,y)$ is $x < y$. 
Write down the interpretation of $\mathscr{A}$ in \textit{I}. 
Is this a true statement or a false one? 
Find another interpretation in which $\mathscr{A}$ is interpreted by a statement with the opposite truth value. \\
(\textit{注意此题有三问: $1)$ 用自然语言(中文/英语)写出 $\mathscr{A}$ 在 $I$下的直观含义;$2)$ 回答在\textit{I}下 $\mathscr{A}$是为\textbf{{\color{purple} 真}}还是为\textbf{{\color{purple} 假}};$3)$ 基于你对第二问的回答,为公式 $\mathscr{A}$ 找一个新的解释,且在这个{\color{purple} 新}解释中,$\mathscr{A}$的真值与你第二问的答案恰好相反})[所以你对第二问的回答很重要]

\noindent\textbf{Answer}:   

(1) The formula $\mathscr{A}$ in $I$ intuitively means that, 

\hspace{6em} \textit{for any integer $x_1, x_2$: if  $(x_1 - x_2) < 0$ then $x_1 < x_2$.}


(2) This interpretation of $\mathscr{A}$ in $I$ is {\color{purple} \textit{true}}. 


(3) Let $D_I$ to be $\mathbb{N}$, $\bar{a}_1$ to be $0$, $\bar{f^2_1} (x, y)$ is $x \times y$, and $\bar{A^2_2}(x,y)$ is $x > y$. 
Clearly, $\mathscr{A}$ is {\color{purple} false} in this new interpretation. 
(other reasonable interpretations are acceptable, of course)
\hfill $\Box$


\vspace{1em}

\dotfill hw-9: feedback
\dotfill

\begin{enumerate}
	\item 对于第三问,当规定了论域 $D_I$ ,一定要小心对 常元 $a_1$ 和 函数符号 $f^2_1(x_1,x_2)$ 的解释是否对论域 $D_I$ {\color{red} \textbf{封闭}}!
	比如:若我们规定 $D_I$ 为所有\textbf{正整数},那么就不能让 $\bar{a}_1 = 0$,因为 $0$ 不是正整数!同理此时不能把 $f^2_1(x_1,x_2)$ 解释为 $x_1 - x_2$,因为正整数不对(通常意义上的)减法封闭!
	我们当然可以重新定义那种对正整数封闭的“减法运算”,不过这就得额外给出明确的形式定义。
	因此,当考虑为一个一阶语言中的公式寻找解释的时候,一定要注意对{\color{purple} \textit{非逻辑符号}}的解释是否\textbf{对论域封闭}的问题。
\end{enumerate}










\vspace{4em}
%\\\\\\\\\\\\\\\\\\\\\\\\\\\\\\\\\\\\\\\\\\\\\\\\\\\\\\\\\\\\\\\\\\\\\\\\\\\\\\\
%\\\\\\\\\\\\\\\\\\\\\\\\\\\\\\\\\\\\\\\\\\\\\\\\\\\\\\\\\\\\\\\\\\\\\\\\\\\\\\\
%=================== hw-10
\noindent\texttt{hw-10 (2023/12/13)}

\emph{p.70}: 22-(a) \quad
Show that none of the following \textit{wfs.} is logically valid.
\[
(a) \qquad 
(\forall x_1) (\exists x_2) A^2_1(x_1,x_2) \to (\exists x_2) (\forall x_1) A^2_1 (x_1,x_2).
\]
\textbf{Proof}:   

It suffices to find an interpretation $I$ such that $I \not \models (\forall x_1) (\exists x_2) A^2_1(x_1,x_2) \to (\exists x_2) (\forall x_1) A^2_1 (x_1,x_2)$.

Let $D_I=\mathbb{N}$ and $\bar{A}^2_1(x,y)$ be `$x < y$'. 


It is clear that the close \textit{wf.} $(\forall x_1) (\exists x_2) A^2_1(x_1,x_2)$ is \textit{true} in this interpretation, 
and the close \textit{wf.} 
$(\exists x_2) (\forall x_1) A^2_1 (x_1,x_2)$ is \textit{false}. 
That is, every valuation satisfies the former and does not satisfy the latter. 
Hence no valuation satisfies the given \textit{wf.} in clause $(a)$. 
It is thus not true in this interpretation and it cannot be logically valid.
\hfill $\Box$


\vspace{1em}
\dotfill hw-10: feedback
\dotfill

\begin{enumerate}
	\item 在找反例的时候,建议大家多找找数学上的例子。日常生活中的很多现象,比如“朋友关系”等都有很大的模糊性,不同人对这些观念的理解可能相差甚大。
\end{enumerate}




%\\\\\\\\\\\\\\\\\\\\\\\\\\\\\\\\\\\\\\\\\\\\\\\\\\\\\\\\\\\\\\\\\\\\\\\\\\\\\\\
\end{document}