% 中文文档类;如果用 article 等文档类,要额外设置中文字体,略麻烦
% 请用  XeLaTeX 编译!!!
\documentclass[UTF8,12pt,a4paper]{ctexart}  

%% 各类数学宏包
\usepackage{amsmath}
\usepackage{amsthm}		%注意:If the amsthm package is used with a non-AMS document class and with the amsmath package, amsthm must be loaded after amsmath, not before.
\usepackage{amssymb}
\usepackage{mathtools}
\usepackage{amsfonts}
\usepackage{wasysym}

%% 设置页边距:
\usepackage{geometry}
\geometry{left=1cm,right=1cm,top=2cm,bottom=2cm}

\usepackage{graphicx} 	%管理图片的宏包

\usepackage{xcolor}

%% 花体字母宏包
\usepackage{mathrsfs}

\usepackage{tcolorbox}
\tcbuselibrary{most}

%% 自定义 「否定」 符号,使之与教材一致
\newcommand{\negs}{\sim\!}


%\\\\\\\\\\\\\\\\\\\\\\\\\\\\\\\\\\\\\\\\\\\\\\\\\\\\\\\\\\\\\\\\\\\\\\\\\\\\\\\
\begin{document}
	

\begin{center}
hw-10 (2023/12/13) \qquad\qquad 姓名:  \hspace{7em}  学号: 
\end{center}

%%%%%%%%%%%%%%%%%%%%%%%%%%%%%%%%%%%%%%%%%%%%%%%%%%%%%%%%%%%%%%%%%%%%%%%%%%%%%%%%
\emph{p.70}: 22-(a) \quad
Show that none of the following \textit{wfs.} is logically valid.
\[
(a) \qquad 
(\forall x_1) (\exists x_2) A^2_1(x_1,x_2) \to (\exists x_2) (\forall x_1) A^2_1 (x_1,x_2).
\]
\textbf{Your answer}:   \hfill (\textit{10 points})


























%\\\\\\\\\\\\\\\\\\\\\\\\\\\\\\\\\\\\\\\
\end{document}