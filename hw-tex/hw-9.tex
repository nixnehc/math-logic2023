% 中文文档类;如果用 article 等文档类,要额外设置中文字体,略麻烦
% 请用  XeLaTeX 编译!!!
\documentclass[UTF8,12pt,a4paper]{ctexart}  

%% 各类数学宏包
\usepackage{amsmath}
\usepackage{amsthm}		%注意:If the amsthm package is used with a non-AMS document class and with the amsmath package, amsthm must be loaded after amsmath, not before.
\usepackage{amssymb}
\usepackage{mathtools}
\usepackage{amsfonts}
\usepackage{wasysym}

%% 设置页边距:
\usepackage{geometry}
\geometry{left=1cm,right=1cm,top=2cm,bottom=2cm}

\usepackage{graphicx} 	%管理图片的宏包

\usepackage{xcolor}

%% 花体字母宏包
\usepackage{mathrsfs}

\usepackage{tcolorbox}
\tcbuselibrary{most}

%% 自定义 「否定」 符号,使之与教材一致
\newcommand{\negs}{\sim\!}


%\\\\\\\\\\\\\\\\\\\\\\\\\\\\\\\\\\\\\\\\\\\\\\\\\\\\\\\\\\\\\\\\\\\\\\\\\\\\\\\
\begin{document}
	

\begin{center}
hw-9 (2023/11/21) \qquad\qquad 姓名:  \hspace{7em}  学号: 
\end{center}

%%%%%%%%%%%%%%%%%%%%%%%%%%%%%%%%%%%%%%%%%%%%%%%%%%%%%%%%%%%%%%%%%%%%%%%%%%%%%%%%
\emph{p.59}: 11 \quad
Let $\mathscr{L}$ be the first order language which includes (besides variables, punctuation, connectives and quantifier) the individual constant $a_1$, the function letter $f^2_1$ and the predicate letter $A^2_2$. Let $\mathscr{A}$ denote the \textit{wf}.
\[
(\forall x_1)(\forall x_2)( A^2_2 ( f^2_1(x_1,x_2), a_1)  \to A^2_2 (x_1,x_2) ).
\]
Define an interpretation $I$ of $\mathscr{A}$ as follows. 
$D_I$ is $\mathbb{Z}$, $\bar{a}_1$ is $0$, $\bar{f^2_1} (x, y)$ is $x-y$, 
$\bar{A^2_2}(x,y)$ is $x < y$. 
Write down the interpretation of $\mathscr{A}$ in \textit{I}. 
Is this a true statement or a false one? 
Find another interpretation in which $\mathscr{A}$ is interpreted by a statement with the opposite truth value.


(\textit{注意此题有三问: $1)$ 用自然语言(中文/英语)写出 $\mathscr{A}$ 在 $I$下的直观含义;$2)$ 回答在\textit{I}下 $\mathscr{A}$是为\textbf{{\color{purple} 真}}还是为\textbf{{\color{purple} 假}};$3)$ 基于你对第二问的回答,为公式 $\mathscr{A}$ 找一个新的解释,且在这个{\color{purple} 新}解释中,$\mathscr{A}$的真值与你第二问的答案恰好相反})[所以你对第二问的回答很重要]

\textbf{Your answer}:   \hfill (\textit{10 points})


























%\\\\\\\\\\\\\\\\\\\\\\\\\\\\\\\\\\\\\\\
\end{document}