% 中文文档类;如果用 article 等文档类,要额外设置中文字体,略麻烦
\documentclass[UTF8,12pt,a4paper]{ctexart}  


%% 各类数学宏包
\usepackage{amsmath}
\usepackage{amsthm}		%注意:If the amsthm package is used with a non-AMS document class and with the amsmath package, amsthm must be loaded after amsmath, not before.
\usepackage{amssymb}
\usepackage{mathtools}
\usepackage{amsfonts}
\usepackage{wasysym}

%% 设置页边距:
\usepackage{geometry}
\geometry{left=1cm,right=1cm,top=2cm,bottom=2cm}


\usepackage{graphicx} 	%管理图片的宏包

\usepackage{xcolor}


%% 花体字母宏包
\usepackage{mathrsfs}

%% 自定义 「否定」 符号,使之与教材一致
\newcommand{\negs}{\sim\!}


%\\\\\\\\\\\\\\\\\\\\\\\\\\\\\\\\\\\\\\\\\\\\\\\\\\\\\\\\\\\\\\\\\\\\\\\\\\\\\\\
\begin{document}
	

\begin{center}
hw-7 (2023/11/07) \qquad\qquad 姓名:  \hspace{7em}  学号: 
\end{center}

%%%%%%%%%%%%%%%%%%%%%%%%%%%%%%%%%%%%%%%%%%%%%%%%%%%%%%%%%%%%%%%%%%%%%%%%%%%%%%%%
\emph{p.49}: 2-(c) \quad
Translate eahc of the following statements into symbols, {\color{purple} first} using no existential quantifiers, and {\color{purple} second} using no universal quantifiers. 

\qquad (c) \hspace{3cm}  \textit{No mouse is heavier than any elephant.}

(\textit{注意:题目要求大家要分别用“全称量词”和“存在量词”符号化句子,因此你的翻译至少有两句})

\textbf{Your answer}:   \hfill ({\color{red} 10 points})































%\\\\\\\\\\\\\\\\\\\\\\\\\\\\\\\\\\\\\\\
\end{document}