% 中文文档类;如果用 article 等文档类,要额外设置中文字体,略麻烦
% 请用  XeLaTeX 编译!!!
\documentclass[UTF8,12pt,a4paper]{ctexart}  

%% 各类数学宏包
\usepackage{amsmath}
\usepackage{amsthm}		%注意:If the amsthm package is used with a non-AMS document class and with the amsmath package, amsthm must be loaded after amsmath, not before.
\usepackage{amssymb}
\usepackage{mathtools}
\usepackage{amsfonts}
\usepackage{wasysym}

%% 设置页边距:
\usepackage{geometry}
\geometry{left=1cm,right=1cm,top=2cm,bottom=2cm}

\usepackage{graphicx} 	%管理图片的宏包

\usepackage{xcolor}

%% 花体字母宏包
\usepackage{mathrsfs}

\usepackage{tcolorbox}
\tcbuselibrary{most}

%% 自定义 「否定」 符号,使之与教材一致
\newcommand{\negs}{\sim\!}


%\\\\\\\\\\\\\\\\\\\\\\\\\\\\\\\\\\\\\\\\\\\\\\\\\\\\\\\\\\\\\\\\\\\\\\\\\\\\\\\
\begin{document}
	

\begin{center}
hw-8 (2023/11/14) \qquad\qquad 姓名:  \hspace{7em}  学号: 
\end{center}

%%%%%%%%%%%%%%%%%%%%%%%%%%%%%%%%%%%%%%%%%%%%%%%%%%%%%%%%%%%%%%%%%%%%%%%%%%%%%%%%
\emph{p.56}: 9-(d) \quad
In each case below, let $\mathscr{A}(x_1)$ be the given \emph{wf}., and let $t$ be the term $f^2_1(x_1,x_3)$. Write out the $wf.$ $\mathscr{A}(t)$ and hence decide in each case whether $t$ is {\color{purple} free for $x_1$} in the given \emph{wf}.
\[
(d) \qquad (\forall x_2) A^3_1 (x_1, f^1_1(x_1),x_2)  \to (\forall x_3)A^1_1(f^2_1(x_1,x_3)).
\]

%===============================================================================
\newtcolorbox{mybox}[2][]
{   colback = white, 
	colframe = black, 
	fonttitle = \bfseries,
	colbacktitle = gray, enhanced,
	attach boxed title to top center={yshift=-2mm},
	title=#2,#1}
\begin{mybox}[colback=white,width=19cm,boxrule=0.2mm]{Recall that}
	
	- {\color{red} $\mathscr{A}(t)$}: if $x_i$ does occur free in $\mathscr{A}(x_1)$, then $\mathscr{A}(t)$ denotes the result of substituting term $t$ for  {\color{purple}every free occurrence} of $x_i$. (cf. \emph{p.54})	
	
	-  $t$ is {\color{red} free} for $x$ in a \emph{wf.} $\phi$: 
	
	\textbf{定义3.11*. (Revised defintion)} \quad
	当一个项 $t$ 可以替换 $\mathscr{A}$ 中变元 $x_i$ 的{\color{purple} 所有自由出现},且不会使得 $t$ 中任何变元与 $\mathscr{A}$ 的其他部分相互作用,我们就称 { \color{red} $t$ 对 $\mathscr{A}$ 中 $x_i$ 是自由的}。
\end{mybox}
%===============================================================================

(\textit{注意此题有两问: 你需要 $1)$ 写出$\mathscr{A}(t)$,且 $2)$ 回答$t$在$\mathscr{A}(x_1)$中是否对$x_1$自由})

\textbf{Your answer}:   \hfill (\textit{10 points})



















%\\\\\\\\\\\\\\\\\\\\\\\\\\\\\\\\\\\\\\\
\end{document}