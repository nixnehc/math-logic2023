% 中文文档类;如果用 article 等文档类,要额外设置中文字体,略麻烦
\documentclass[UTF8,12pt,a4paper]{ctexart}  


%% 各类数学宏包
\usepackage{amsmath}
\usepackage{amsthm}		%注意:If the amsthm package is used with a non-AMS document class and with the amsmath package, amsthm must be loaded after amsmath, not before.
\usepackage{amssymb}
\usepackage{mathtools}
\usepackage{amsfonts}
\usepackage{wasysym}

%% 设置页边距:
\usepackage{geometry}
\geometry{left=1cm,right=1cm,top=2cm,bottom=2cm}


\usepackage{graphicx} 	%管理图片的宏包

\usepackage{xcolor}


%% 花体字母宏包
\usepackage{mathrsfs}

%% 自定义 「否定」 符号,使之与教材一致
\newcommand{\negs}{\sim\!}


%\\\\\\\\\\\\\\\\\\\\\\\\\\\\\\\\\\\\\\\\\\\\\\\\\\\\\\\\\\\\\\\\\\\\\\\\\\\\\\\
\begin{document}
	

\begin{center}
hw-5 (2023/10/17) \qquad\qquad 姓名:  \hspace{7em}  学号: 
\end{center}

%%%%%%%%%%%%%%%%%%%%%%%%%%%%%%%%%%%%%%%%%%%%%%%%%%%%%%%%%%%%%%%%%%%%%%%%%%%%%%%%
\emph{p.36}: 1-(c) \quad
Write out proofs in $L$ for the following $wfs$.
\[
(c) \qquad  (p_1 \to (p_1 \to p_2)) \to (p_1 \to p_2)
\] 


\textbf{Your proof}:












%%%%%%%%%%%%%%%%%%%%%%%%%%%%%%%%%%%%%%%%%%%%%%%%%%%%%%%%%%%%%%%%%%%%%%%%%%%%%%%%
\vspace{12cm}  % 如果用 latex 答题 请把这个垂直间距删除,或调整为自己喜欢的间距


\emph{p.37}: 5 \quad
The rule $HS$ is an example of a legitimate additional rule of deduction for $L$. Is the following rule legitimate in the same sense: from the $wfs.$ $\mathscr{B}$ and $(\mathscr{A} \to (\mathscr{B} \to \mathscr{C}))$, deduce $(\mathscr{A} \to \mathscr{C})$ ?


\textbf{Your answer}:





















%\\\\\\\\\\\\\\\\\\\\\\\\\\\\\\\\\\\\\\\
\end{document}